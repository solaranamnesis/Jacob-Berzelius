\documentclass[a4paper, 11pt, oneside]{article}
\usepackage[utf8]{inputenc}
\usepackage[T1]{fontenc}
\usepackage[ngerman]{babel}
\usepackage{fbb} %Derived from Cardo, provides a Bembo-like font family in otf and pfb format plus LaTeX font support files
\usepackage{booktabs}
\setlength{\emergencystretch}{15pt}
\usepackage{fancyhdr}
\usepackage{graphicx}
\graphicspath{ {./} }
\usepackage{microtype}
\usepackage[figurename=]{caption}
\begin{document}
\begin{titlepage} % Suppresses headers and footers on the title page
	\centering % Centre everything on the title page
	%\scshape % Use small caps for all text on the title page

	%------------------------------------------------
	%	Title
	%------------------------------------------------
	
	\rule{\textwidth}{1.6pt}\vspace*{-\baselineskip}\vspace*{2pt} % Thick horizontal rule
	\rule{\textwidth}{0.4pt} % Thin horizontal rule
	
	\vspace{1\baselineskip} % Whitespace above the title
	
	{\scshape\LARGE Über Meteorsteine\\[1.25pt]}
	
	\vspace{1\baselineskip} % Whitespace above the title

	\rule{\textwidth}{0.4pt}\vspace*{-\baselineskip}\vspace{3.2pt} % Thin horizontal rule
	\rule{\textwidth}{1.6pt} % Thick horizontal rule
	
	\vspace{1\baselineskip} % Whitespace after the title block
	
	%------------------------------------------------
	%	Subtitle
	%------------------------------------------------
	
	{\scshape Kongl. Vetensk. Acad. Handl. f. 1834.} % Subtitle or further description
	
	\vspace*{1\baselineskip} % Whitespace under the subtitle
	
    {\scshape\small Von J. J. Berzelius.} % Subtitle or further description
    
	%------------------------------------------------
	%	Editor(s)
	%------------------------------------------------
    \vspace*{\fill}

	\vspace{1\baselineskip}

	{\small\scshape Leipzig 1834.}
	
	{\small\scshape{Verlag von Johann Ambrosius Barth.}}
	
	\vspace{0.5\baselineskip} % Whitespace after the title block

    \scshape Internet Archive Online Edition  % Publication year
	
	{\scshape\small Namensnennung Nicht-kommerziell Weitergabe unter gleichen Bedingungen 4.0 International} % Publisher
\end{titlepage}
\setlength{\parskip}{1mm plus1mm minus1mm}
\clearpage
\tableofcontents
\clearpage
\section*{}
\paragraph{}
Erst seit Anfang dieses Jahrhunderts hat man es als wissenschaftlich bewiesen angesehen, dass von Zeit zu Zeit größere und kleinere Steinmassen auf die Erde herabfallen, gewöhnlich begleitet von einem stark krachenden donnerähnlich rollenden Getöse und einer Feuererscheinung, wobei der Stein auf seiner Oberfläche so schnell verglast wird, dass sein Inneres vor der verändernden Einwirkung der Hitze geschützt bleibt. Gewöhnlich zerspringt dadurch der Stein während seines Falls und die Stücke werden ziemlich weit umbergeschleudert. Wiewohl aus älterer und neuerer Zeit dergleichen Steinfälle beschrieben worden sind, so glaubte der besonnene Naturforscher doch lange die Zuverlässigkeit solcher Nachrichten in Zweifel ziehen zu müssen, da kein annehmlicher Grund zu der Vermutung vorhanden war, den Ursprung so schwerer Körper aus der Atmosphäre abzuleiten. Die sicherere Kenntnis, welche wir gegenwärtig zu besitzen glauben, ward begründet durch einen am 13. Dezember 1795 in England, zu Woldcottage in Yorkshire, sich ereignenden: und gehörig beglaubigten Meteorsteinfall. Howard, der einige Jahre darauf eine Untersuchung dieses und mehrere anderer, angeblich vom Himmel gefallener Steine vornahm, fand sie im Ansehen und in der Zusammensetzung übereinstimmend, dagegen bestimmt verschieden von den Mineralien irdischer Abkunft. Als hauptsächlichstes Kennzeichen fand er, dass sie metallisches Eisen eingesprengt enthielten und dass dies Eisen nickelhaltig war. Howard teilte seine Untersuchung i. J. 1802 der Königl. Gesellschaft in London mit. Sie erregte allgemeine Aufmerksamkeit, wiewohl das von Howard aus seiner Untersuchung gezogene Resultat, welches von Pictet der französischen Akademie der Wissenschaften mitgeteilt wurde, in der ersten Zeit für einen Irrtum gehalten wurde. Der Zufall fügte es jedoch, dass sich wenige Monate darauf, am 26. Apr. 1803, zu L'Aigle im. Dép. de l’Orne einer der größten und merkwürdigsten Steinregen ereignete, wobei auf eine gewisse Fläche gegen ein Paar tausend Steinstücke ausgesäte wurden. Die Zahl der Augenzeugen war groß, und die französische Akademie der Wissenschaften, schon aufmerksam geworden auf solche Ereignisse, übertrug ihrem Mitgliede Biot eine Untersuchung der Verhältnisse an Ort und Stelle. Sein Bericht hob allen noch übrig gebliebenen Zweifel, dass die Steine von oben herabgefallen waren, und zwar unter Erscheinungen, die mit den von früheren Meteorsteinfällen angegebenen eine solche Ähnlichkeit hatten, dass auch diese dadurch an Glaubwürdigkeit gewannen.

Nun fing man an darüber nachzudenken, wo etwa diese fallenden Körper entsprungen sein mochten. Die Vermutung, dass sie Auswürflinge irdischer Vulkane seien, bewährte sich nicht, sowohl wegen der großen Entfernung der Orte des Falls von Vulkanen als auch wegen der Verschiedenartigkeit der gewöhnlichen vulkanischen Produkte. und der Meteorsteine. Man hat sie im Ernste als gebildet aus den Bestandteilen der Luft angesehen; allein wir wissen weder, ob die Bestandteile der Meteorsteine in Luftform existieren können, noch, ob sie aus den gewöhnlichen Bestandteilen der Luft zusammengesetzt seien; und überdies, wenn dies auch der Fall wäre, haben doch mehre Meteorsteine eine se große Masse gehabt, dass ihre Bildung im der Atmosphäre unmöglich in der kurzen Zeit ihres Falles durch die Luft vor sich gehen konnte, besonders da notwendigerweise der Fall schon bei Absetzung des ersten festen Teilchens hätte beginnen müssen.

Anaxagoras vermutete von einem zu seiner Zeit bei Aegos Potamos gefallenen Steine, dass er von einem anderen Weltkörper ausgeworfen worden sei. Diese Ansicht schließt vermutlich die Wahrheit ein, und ist auch durch die Forschungen unserer Zeit unterstützt worden. Olbers äußerte i. J. 1795 in einem Bericht über den am 16. Juli 1794 zu Siena in Italien geschehenen Meteorsteinfall die Idee, dass dergleichen Steine vom Monde ausgeworfen sein könnten, hielt es aber doch für wahrscheinlicher, dass sie aus dem Vesuv herstammten. Im Jahr 1802 sprach Laplace, auf Veranlassung der Arbeit von Howard, dieselbe Idee aus, mit dem Zusatz, die Feuererscheinung entspringe aus der Zusammendrückung der Luft, in Folge der unendlichen Geschwindigkeit, mit welcher der Meteorstein in die Atmosphäre eindringe, welche aber durch den Widerstand der Luft so verringert werde, dass der Fall zuletzt nur mit der gewöhnlichen Fallgeschwindigkeit geschehe.

Die uns zugewandte Seite des Mondes ist voller Höhen, und darunter finden sich viele Berge, die den mit Kratern versehenen Vulkanen unserer Erde ganz ähnlich gebildet sind, und dabei so große Dimensionen haben, dass man mit guten Fernröhren in die Krater sehen und sehr wohl unterscheiden kann, dass die eine Hälfte der Innenseite von der Sonne beleuchtet und die andere beschattet ist, während der Ring, welcher den Krater bildet, hervorsteht. Dies lässt vermuten, dass diese Berge ihre Form durch dieselbe Ursache wie die auf der Erde erhalten haben, d. h. durch Eruptionen. Wenn aber die Kraft, welche auf dem Monde Eruptionen hervorbringt, eben so groß ist als die Wurfkraft der irdischen Vulkane, so müssen sich die geworfenen Körper bedeutend weiter von dem Monde entfernen als von der Erde; denn erstlich ist die Masse des Mondes nur 1,45 Prozent von der der Erde, und damit steht auch die Schwere auf dem Mond im Verhältnis; zweitens hat der Mond keinen Luftkreis, oder wenigstens einen: so lockeren, dass bei Fixsternbedeckungen durch den Mond keine Strahlenbrechung darin ‚wahrnehmbar ist. Der Auswurf geschieht folglich in einen luftleeren Raum, ohne einen solchen mechanischen Widerstand für die Bewegung der geworfenen Körper wie ihn die Atmosphäre der Erde darbietet, wo der Körper daher. bald zur Ruhe kommt. Wenn drittens der Auswurf gegen die Erde gerichtet ist; so nimmt die Anziehung der Erde zu dem geworfenen Körper beständig zu, während die des Mondes stetig abnimmt. Und viertens liegt die Gleichgewichtsgränze zwischen der Erde und dem Monde bedeutend näher am letzteren. Biot gibt an, dass eine Wurfkraft von 7771 Pariser Fuß in der Sekunde diese Grenze erreiche; mit einem geringen Kraftüberschuss wird der Körper dieselbe übersteigen und dann auf die Erde fallen müssen. Diese Geschwindigkeit ist fünf bis sechs Mal grösser als die einer 24 pfündigen Kanonenkugel beim Austritt aus der Kanone, und wird von der Wurfkraft unserer Vulkane übertroffen.\footnote{Laplace in v. Zachs Monatl. Korrespondenz, 1802 Septemb. S. 277.} Die Berechnungen, welche sowohl Olbers\footnote{Gilberts Annal. d. Physik, Bd. XV, S. 39.} als Poisson\footnote{Ebendaselbst, S. 329.} hierüber angestellt haben, zeigen, dass die Idee eine physische Möglichkeit einschließe.

Verschiedene Umstände bei den Meteorsteinen passen wohl zu dem, was wir glauben von dem Monde zu wissen. Die Meteorsteine sind durchsetzt mit metallischem Eisen, welches, wenn der Stein mit lufthaltigem Wasser befeuchtet wird, allmählig zu Eisenoxydhydrat rostet wie es unter gleichen Umständen mit den Mineralien der Erdkruste der Fall ist. In ihrer ursprünglichen Lagerstätte mangelt also Luft, oder beides, Luft und Wasser. Auch haben astronomische Untersuchungen keine Spur von so großen Wasseransammlungen auf dem Mende gefunden, dass sie mit guten Fernröhren zu entdecken wären. Ich weiß nicht, dass man in den Meteorsteinen chemisch gebundenes Wasser gefunden habe. - Wir werden in der weiterhin folgenden Untersuchung finden, dass die meisten Meteorsteine einander in der Zusammensetzung so ähnlich sind, dass man sie als von demselben Berg herrührend ansehen kann, während nur wenige von abweichender Beschaffenheit gefunden wurden. Soweit es zulässig ist, aus den Verhältnissen auf der Erde einen Schluss zu ziehen, kann man die übrigen Weltkörper auch gar nicht als homogene Gemenge von Mineralien ansehen, vielmehr hat die Geschichte ihrer ursprünglichen Bildung sicher viele Ähnlichkeit mit der Geschichte der der Erde. Die Felsarten aus verschiedenen Gegenden eines anderen Weltkörpers werden also in der Zusammensetzung verschieden sein können.

Der Mond kehrt der Erde beständig dieselbe Seite zu. Der Mittelpunkt seiner siebtbaren Scheibe macht folglich deren beständig uns zugewandten Gipfel aus, dessen Eruptionen ihre Projektile am leichtesten über die Gleichgewichtslinie hinauswerfen, und folglich müssen die auf die Erde fallenden Meteorsteine, angenommen, dass sie vom Monde kommen, in größter Zahl von hier ab ausgeworfen worden sein. Sie können folglich einem ganz beschränkten Gebirgszug angehören, und dann lässt sich ihre große Gleichheit im Ansehen und in der Zusammensetzung leicht begreifen. Die Auswürflinge von Eruptionen, welche seitwärts dieses Gipfels geschehen, fliegen in einer nicht mehr direkt gegen die Erde gerichteten Linie fort, und müssen also seltener in den Anziehungskreis der Erde gelangen. Wenn die Bergarten dieser Gegenden verschieden sind von denen auf dem Gipfel der uns zugewandten Mondeshälfte, so sieht man leicht ein, dass uns von daher Meteorsteine von anderer als der gewöhnlichen Beschaffenheit zukommen müssen, zugleich aber auch, dass dies vergleichungsweise selten geschehen müsse. Darf man annehmen, dass der uns zugewandte Mondsscheitel so mit Nickeleisen ‚durchsetzt ist als es die Meteorsteine sind, und dass die übrigen Teile, oder wenigstens die beständig von’ der Erde abgewandte Halbkugel, wenig oder gar nichts davon enthalten, so würde daraus folgen, dass der Mond, wenn auf ihn die Erde, außer ihrer allgemeinen, von der Schwere herrührenden Anziehung, noch eine magnetische Anziehung ausübte, den eisenreichsten Teil seiner’ Kugel gegen die Erde wenden müsse, und dass daraus die wunderbare Erscheinung entstehe, dass der Mond uns unverwandt die nämliche Seite zukehrt.

Es ist jedoch auch möglich, dass die Meteorsteine von einem anderen kosmischen Orte herkommen. Olbers äußerte bekanntlich die Vermutung, dass die kleinen Planeten zwischen Mars und Jupiter Stücke eines zersprungenen Planeten sein könnten, in Folge weicher Vermutung mehre dergleichen Stücke. gesucht wurden, und Olbers selbst eines derselben fand. Wem eine solche Katastrophe stattfand, was durch den bedeutenden Winkel, welchen die Bahn der Pallas mit den Bahnen der übrigen Planeten macht, bestätigt zu werden scheint, so muss eine unendliche Menge kleiner Stücke umhergeschleudert worden sein in Richtungen, dass sie um die Sonne abnehmende Bahnen beschreiben, wodurch sie dann leicht auf ihrem Wege in die Attraktionssphäre anderer Planeten geraten und auf sie niederfallen. Man hat auch vermutet, die Materie des Weltalls befinde sich zum Teil in einer noch nicht geordneten Bewegung und die Meteorsteine seien solche mehr oder weniger große Massen‚ welche zuweilen in die Attraktionssphäre der Erde geraten; allein diese Vermutung ist von allen die wenigste wahrscheinliche. Das Weltsystem scheint von der bestimmtesten Ordnung zu zeugen, und überdies wird ‚nach dieser Vermutung die identische Beschaffenheit der Meteormassen noch weniger begreiflich.

Indes lässt sich als ausgemacht ansehen, dass sie nicht von der Erde, sondern von einem anderen Weltkörper herstammen, und folglich die Beschaffenheit der außerhalb der Erde vorkommenden wägbaren Stoffe verkünden. In dieser Beziehung haben die Meteorsteine ein außerordentliches Interesse. Von Wichtigkeit dabei ist es, nicht nur die Mineralien, aus denen sie bestehen, auszumitteln, sondern auch die geringste Spur von zuvor noch nicht darin gefundenen Elementen. Möglich wäre es, darunter solche zu finden, welche noch nicht auf der Erde angetroffen sind.

Wie stark auch die Vermutung im Voraus war, dass die Schwerkraft im ganzen Universum herrsche, so haben doch die Astronomen mit besonderem Interesse in den Umläufen der Doppelsterne umeinander eine Wirkung derselben Gravitationsgesetze erkannt, welche für unser Planetensystem gültig sind. Nicht minder interessant ist es zu erfahren, aus welchen Stoffen andere Weltkörper bestehen, und die Gewissheit zu erlangen, dass sie von einerlei Natur mit denen sind, welche die Masse der Erde ausmachen. Haben wir gleich die letzteren noch nicht alle in den Metvorsteinen aufgefunden, so haben wir doch einen großen Teil der allgemeiner verbreiteten darin angetroffen, und wir werden in dem Folgenden sehen, dass es geglückt ist zu bestimmen, in welchen chemischen Verbindungen sie darin enthalten sind.

Die Arbeit über Meteorsteine, welche ich hier die Ehre habe der K. Akademie zu überreichen, ist veranlasst worden durch eine Aufforderung, den am 25. November 1833, um 6 1/4 Uhr Abends in der Nachbarschaft von Blansko in Mähren niedergefallenen Meteorstein chemisch zu untersuchen. Er bildete wie gewöhnlich ein stark leuchtendes Feuerphänomen und seinem Falle ging ein donnerähnliches Getöse voran. Der Bergamts-Direktor Reichenbach, welcher sich damals auf dem Felde befand und Zeuge des Meteors war, stellte hernach mit ‚einer starken Mannschaft eine Aufsuchung der gefallenen Masse an, und dadurch glückte es endlich, kleine Stücke, zum Belauf von etwa einem halben Pfunde aufzufinden, aber die Hauptmasse zu entdecken gelang, hauptsächlich wegen, der waldigen Beschaffenheit dieser Gegend, noch nicht.

\section{Meteorstein von Blansko.}
\paragraph{}
Dieser Meteorstein gehört zu den häufigste vorkommenden, und kann, neben einem derselben gelegt, z. B. neben den von Benares, l’Aigle, Berlougville, u. s. w., von ihm nicht unterschieden werden. Seine Beschreibung ist folglich die Beschreibung von diesen. Er hat die gewöhnliche, äußerlich geschmolzene Rinde, eine hellgraue, etwas rostfleckige, feinkörnige Bruchfläche, die hie und da runde Kügelchen von gleicher Farbe mit dem Steine zeigt; letztere können ausgelöst werden und hinterlassen dann eine glatte Höhlung. Er enthält viel Nickeleisen, und sehr wenig Schwefeleisen, in feinen Partien überall eingesprengt, und dadurch zeigt er glänzende Punkte, von denen einige in einer gewissen Richtung rötlich erscheinen, indes doch nichts anderes sind als angelaufenes Nickeleisen. Zerstößt man den Stein zu einem gröblichen Pulver, so kann das Nickeleisen mit einem Magneten ausgezogen, und unter Wasser von der sichtlich anhängenden Steinsubstanz abgewaschen werden, so dass die Eisenteilchen fast silberweiß zurückbleiben. Indes schließen dieselben doch in ihren Vertiefungen und Höhlungen noch viel Steinsubstanz ein, welche bei Auflösung des Eisens teils zersetzt, teils abgelagert wird. In dem zu meinen Versuchen angewandten Stücke waren 17,15 Prozent Nickeleisen, von denen die eingeschlossene Steinmasse bereits abgerechnet ist. Unter einem zusammengesetzten Mikroskop kann man mit Deutlichkeit keine anderen Bestandteile unterscheiden als ein weißes splittriges Mineral, welches scheint durchscheinend zu sein und bei den Röstflecken gelblich ist, und die metallischen kantigen Körner. Dasselbe ist der Fall, wenn man das gröbliche Pulver des Minerals unter dem Mikroskop betrachtet; allein dann sind seine Teile durchsichtiger.

Auf mechanischem Wege habe ich aus dem Meteorstein nichts anderes abscheiden können als das weiße Mineral, die runden Kügelchen und Nickeleisen. Auf chemischen Wege habe ich abgeschieden: Nickeleisen, Schwefeleisen, Chromeisen, ein weißes Mineral, welches von Säuren zerlegt wird, und ein ähnliches, welches von Säuren nicht angegriffen wird. Obgleich der Stein ziemlich gleichförmig gemengt zu sein scheint, so ist doch ganz deutlich das Nickeleisen an gewissen Stellen reichlicher als an andern zugegen. Gewisse Teile des Steins geben beim Reiben ein dunkleres Pulver als andere.

Dor nicht magnetische Teil des Steins verhält sich vor dem Lötrohr folgendermaßen: Er gibt, gelinde geglüht, kein Wasser und verändert sich nicht. Wird ein Stück an offener Luft gebrannt, so ist der Geruch nach schwefliger Säure erkennbar, und der Stein wird obenauf schwarz, inwendig rotfleckig. Das Pulver des Steins brennt sich im Glühen rot, und schmilzt endlich, aber weit träger als Feldspat, zu einer schwarzen Glaskugel mit matter Oberfläche, ganz ähnlich der schwarzen Rinde, welche den Stein von außen umgibt. Von Borax wird er leicht zu einem eisengrünen Glase gelöst; ebenso vom Phosphorsalz, jedoch mit Hinterlassung eines Kieselskeletts. Mit kohlensaunen Natron schmilzt der Stein zu einer schwarzen Kugel. Dies ist das, gewöhnliche Verhalten der Meteorsteine vor dem Lötrohr.

Ich werde diese Untersuchung in zwei Hauptabschnitte teilen, nämlich 1. von den nicht magnetischen, und 2. von den mit dem Magneten ausziehbaren Teilen handeln.

Diese mechanische Abscheidung durch den Magneten scheint zwar ganz leicht zu sein lässt sich aber doch so gut wie unmöglich ganz vollständig bewirken. Das Schwefeleisen verwandelt sich beim Reiben in Pulver, welches sich unterschiedslos dem Steinpulver beimengt und ihm eine dunklere Farbe erteilt. Um erst das meiste auszuziehen stieß man den Stein zu grobem Pulver, und zog aus diesem das Magnetische unter Wasser aus. Als dem Magneten nichts mehr folgen wollte, rieb man den Rückstand zu feinem Pulver und behandelte dasselbe abermals unter Wasser mit dem Magneten, wodurch an wieder eine kleine Portion magnetischer Teile erhielt. Den Rückstand zerrieb man nun in einer Porphyrschale und schlemmte ihn. Das trockne Pulver war hellgrau und rot, bei Übergießung mit Salzsäure, nach Schwefelwasserstoffgas, und, beim Glühen, nach schwefliger Säure, beide Mal schnell vorübergehend, aber doch die Gegenwart einer Portion Schwefeleisen beweisend, die vom Magneten nicht ausgezogen worden war.

Zur Vermeidung unnötiger Weitläufigkeiten werde ich ein für alle Mal den beiden Analysen eingeschlagenen Weg beschreiben und sodann, bei jeder einzelne Art, wo nicht von diesem Wege abgewichen wurde, nur das Resultat anführen.

A. Das Steinpulver wurde in einem Platingefäß mit konzentrierter Salzsäure zersetzt; es entstand dadurch eine Gelatinirung, die aber doch nur partiell war. Während der Zersetzung war das Gefäß mit einem reinen Uhrglase bedeckt; dies wurde aber nicht angegriffen, zeigte also die Abwesenheit von Fluorverbindungen an. Die Masse wurde eingetrocknet, mit Salzsäure befeuchtet und nach einer Weile mit Wasser ausgezogen. Das Ungelöste wurde ausgewaschen, noch feucht zwei Mal mit kohlensaurem Natron gekocht, die Lösung jedesmal mit vielem kochenden Wasser verdünnt und dann noch siedend heiß filtriert. Das Gewicht des nun Ungelösten gab den Gehalt des Meteorsteins an in Säuren unlöslichen Verbindungen, und, durch Subtraktion von dem Gewicht der angewandten Menge, auch die Menge dies durch Säuren zerlegten Teils des Minerals. Die Lösung in kohlensaurem Natron wurde mit Salzsäure übersättigt und im Wasserbade zur Trockne verdunstet, bei Wiederauflösung in Wasser blieb die Kieselerde des von der Säure zersetzten Minerals zurück. Die Lösung in Wasser wurde mit Ammoniak geprüft, dass sie keinen Niederschlag gab, und das Waschwasser von der Kieselerde wurde zur Trockne verdunstet, worauf der Rückstand, bei Behandlung mit Wasser, noch etwas Kieselerde hinterliefs, welche das Wasser während des Auswaschens aufgenommen hatte.

B. Die Lösung: des zersetzten Minerals in Salzsäure wurde mit Salpetersäure oxydiert, die Lösung mit ätzendem Ammoniak gefällt, um in der Flüssigkeit neben einem Teil der Talkerde, Kalk und Alkali zurückzuhalten. Als ich nun die Flüssigkeit sogleich mit kohlesaurem Ammoniak niederschlage, erhielt ich immer weniger Kalk als wirklich im Stein vorhanden war. Ich fand bemach, aber zu spät, um noch Gebrauch davon machen zu können, dass der Plan der Analyse fehlerhaft war, da nämlich der Meteorstein Zinnoxid enthielt, weiches vor der Oxydation mit Salpetersäure hätte durch Schwefelwasserstoff gefällt werden müssen. Indes ist die Menge der Zinnoxyds so gering, dass es ganz vernachlässigt werden kann, nachdem man weiß, dass es sich darin befindet.

Die mit ätzendem Ammoniak gefällte Flüssigkeit wurde mit etwas Schwefelwasserstoff-Schwefelammonium versetzt (wodurch sie schwarz ward), und damit in einer verkorkten Flasche stehen gelassen, bis sie, mit einem Stich ins Gelbe, klar geworden war. So lange die klar gewordene Flüssigkeit farblos ist, kenn man nicht sicher sein den ganzen Nickelgehalt ausgefällt zu haben. Zur Klärung sind oft 24 Stunden erforderlich. Diese Methode zur Abscheidung des Nickeloxyde ist die beste, welche ich kenne. Indes hat sie doch zwei Fehler. Der eine besteht darin, dass beim Waschen etwas schwefelsaures Nickeloxyd wieder gebildet wird, und der andere, dass durch wechselseitige Verwandtschaft etwas Schwefelmagnesium entsteht und sich mit dem Schwefelnickel niederschlägt, besonders wenn das Gemenge zum Klären in die Wärme gestellt wird. Indes haben die Fehler der Methode keinen wesentlichen Einfluss auf das Resultat der Analyse Um aus dem Schwefelmetall die Menge des Nickeloxyds zu bestimmen, ward es geröstet, in Salzsäure gelöst, mit ätzendem Kali gefällt und gewaschen, gewägt und geglüht. Das so erhaltene Nickeloxyd enthielt bei allen Versuchen Kupferoxyd, wie es sich vor dem Lötrohr durch die gewöhnliche Reduktion zu Kupferoxydul nachweisen ließ. Es zeigte noch ein anderes Verhalten, welches meine Aufmerksamkeit erregte; es gab nämlich, eingeschmolzen in Phosphorsalz und mit der Oxydationsflamme beblasen, ein Glas, welches keim Erkalten undurchsichtig und farblos ward. Die Veranlassung davon war, wie es sodann zeigte, ein Gehalt von Zinnoxid. Wird das Nickeloxyd mit Phosphorsalz geschmolzen, metallisches Zinn hinzugesetzt und dann so stark darauf geblasen, dass sowohl das Nickel als das Kupfer im reduzierten Zustand vom Zinn aufgenommen wird, so erhält man beim Erkalten ein trübes Glas von blassblauer Farbe, was einen geringen Gehalt von Kobalt im Nickeloxyd andeutet. Nachdem das Zinn als beständiger Bestandteil der Meteorsteine aufgefunden worden, änderte ich die Operationsmethode dahin ab, dass ich die Lösung in Salzsäure erst mit Schwefelwasserstoffgas füllte, den Überschuss desselben durch Abdunsten der filtrierten Flüssigkeit entfernte und darauf die Flüssigkeit im konzentrierten Zustand zum Behufe der Oxydation des Eisens auf die im Übrigen zu Anfange von B angeführte Weise mit etwas Salpetersäure vermischte.

C. Die mit Schwefelwasserstoff-Schwefelammonium gefällte Flüssigkeit wurde mit oxalsauren Ammoniak auf Kalk geprüft, gewöhnlich aber von diesem nicht die geringste Spur erhalten. Da nach mehreren Stunden keine Trübung bemerkt wurde, dunstete ich die Flüssigkeit im Wasserbade zur Trockne ein, erhitzte den Rückstand vorsichtig in einer Porzellanschale bis zur Zersetzung des salpetersauren Ammoniaks, dann bis zur Verjagung des Salmiaks, und nun über einer Weingeistflamme bis zam gelinden Glühen der Schale, solange noch ein Geruch nach Salzsäure verspürt werden konnte. Nach dem Erkalten der Schale wurde die Masse mit ätzendem Ammoniak befeuchtet, mit Wasser ausgezogen und die Talkerde auf ein Philtrum gebracht. Gewöhnlich ward sie, in Folge eines Mangangehalts, beim Glühen rosenrot, Die Lösung wurde im Platintiegel zur Trockne verdunstet, der Salmiak verjagt und der Boden des Tiegels bis zum anfangenden Glühen erhitzt, dann ein zu einer Kugel aufgerolltes und mit destilliertem Wasser befeuchtetes Philtrum hineingeworfen und der Deckel aufgelegt. Der Zweck hierbei war, das rückständige Chlormagnesium in einer Atmosphäre von Wassergas zu erhitzen, um allen Chlorgehalt völlig fortzunehmen. Nachdem das Papier sich verkohlt hatte, ließ ich den Tiegel erkalten, nahm die Papierkohle heraus, oder verbrannte sie, wenn sie festgeklebt war. Das Wasser zog nun Chlor-Alkalium aus, welches, nach Verdunstung des Wassers zur Trockne, gewogen wurde. Durch Zusatz von Platinchlorid, Abdunstung des Salzes und Behandlung desselben mit Alkohol wurde der Gehalt an Chlorkalium darin auf die gewöhnliche Weise bestimmt. Die im Tiegel festsitzende, von Kohle geschwärzte Talkerde wurde weiß gebrannt und gewägt.

D. Das in B mit ätzendem Ammoniak Gefällte wurde in Salzsäure gelöst und mit kohlensaurem Ammoniak niedergeschlagen, weil ich fand, dass ätzendes Kali keine Tonerde daraus zog, ehe diese neue Fällung vor sich gegangen war. Nun blieb sehr viel Talkerde in der Losung zurück, aus der sie auf die gewöhnliche Weise erhalten wurde. Sie enthielt gewöhnlich eine geringe Spur von Nickeloxyd.

Aus dem mit kohlensaurem Ammoniak erhaltenen Niederschlag zog ätzendes Kali Tonerde aus, doch immer nur sehr wenig, welche dann auf gewöhnliche Weise abgeschieden wurde.

Der so behandelte Rückstand wurde in Salzsäure gelöst und bei Digestion auf einem Wasserbade mit bernsteinsaurem Ammoniak gefällt, wodurch Eisenoxyd abgeschieden wurde.

Aus der so gefällten Flüssigkeit wurde mit kohlensaurem Kali, nach Zersetzung der Ammoniaksalze und Verjagung des Ammoniaks, Nickeloxyd erhalten, welches Talkerde und etwas Manganoxyd enthielten. Talkerde und Nickeloxyd zu trennen ist äußerst schwer, und lässt sich unmöglich mit vollständiger Genauigkeit bewirken. Nachdem ich gefunden, dass oxalsaures Ammoniak, so wie das Lösen in Essigsäure und Behandeln der durch Ammoniak neutralisierten Lösung mit Schwefelwasserstoffgas kein genügendes Resultat lieferte, bediente ich wich der Digestion des geglühten Oxyds mit verdünnter Salpetersäure, wobei das Oxyd meist ungelöst zurückblieb, füllte die Lösung mit Schwefelwasserstoff-Schwefelammonium, filtrierte sie, trocknete sie ein, und brannte die Salpetersäure fort, um die Talkerde zu erhalten, deren Gewicht von dem gemeinschaftlichen des Nickeloxyds und der Talkerde abgezogen wurde. Nickelfrei wurde zwar die Talkerde auf diese Weise nicht erhalten; allein die Spur, welche sie von diesem zurückhielt, hatte keinen wesentlichen Einfluss auf das Resultat. Die Talkerde ist nach dem Glühen von einem andern gebrannten weißen Rückstand daran zu erkennen, dass sie beim geröteten Lackmuspapier die blaue Farbe wieder herstellt.

E. Der in A unlösliche Teil des Meteorsteins, welches erst mit Salzsäure und dann kochend mit kohlensaurem Natron behandelt worden, wurde bei verschiedenen Versuchen auf dreierlei Weisen behandelt, nämlich entweder mit kohlensaurem Baryt oder kohlensaurem Natron geglüht, oder auch mit Fluorwasserstoffsäure behandelt.

Glühen mit kohlensaurem Baryt. Die geglühte Masse, weiche der starken Hitze eines Kohlenofens ausgesetzt worden, war nicht geschmolzen. Ihre Farbe war grau geworden. Sie gelatinierte wie gewöhnlich mit Salzsäure. Die Kieselerde wer im feuchten Zustand dunkelgrau, im trocknen aber weiß. Nach Auflösung durch Kochen mit kohlensaurem Natron blieb ein schwarze Pulver zurück, welches sich nicht weiter lösen wollte, und nach dem Trocknen braun wurde. Dieses Pulver ergab sich vor dem Lötrohr als Chromeisen, welches in dieser analytischen Methode eine Zersetzung erlitt; als es aber in Phosphorsalz aufgelöst wurde, zeigte es zwei Eigenheiten, nämlich dass metallisches Platin aus der Oberfläche der Kugel herauskroch und dass die klare Kugel beim Erkalten trübe grün ward. Das Platin rührte sichtlich im Versuche von dem Platintiegel her, welcher bei dem Glühen schien angegriffen worden zu sein. Ich würde dieses Umstandes gar nicht erwähnt haben, wenn ich nicht einen ganz gleichen bei der Zerlegung des Minerals durch Fluorwasserstoffsäure beobachtet hätte, wiewohl hier kein Anfressen des Tiegels merkbar war. Indes habe ich doch keinen Grund, dieses Platin von etwas anderem als dem gebrauchten Platintiegel herzuleiten. Den Bestandteil, welcher die Phosphorsalzkugel trübte, stellte ich auf folgende Weise dar. Das Chromeisen wurde durch Schmelzen mit saurem schwefelsauren Kali aufgelöst, wobei das Platin obenauf floss und einen Teil der Oberfläche versilberte, wo es sich während des Abkühlens erhielt und abgenommen werden konnte. Die Masse wurde in Wasser gelöst und durch die schwach grüne Lösung ein Strom von Schwefelwasserstoffgas geleitet; es entstand dadurch ein gelbbrauner Niederschlag, welcher, nach dem Rösten, mit kohlensaurem Natron und Borax im Reduktionsfeuer behandelt, ein geschmeidigeres Zinnkorn gab. Mit Phosphorsalz konnte darin ein geringer Kupfergehalt entdeckt werden. Das schwarze Pulver war also Chromeisen, welches eine verhältnismäßig sehr geringe Menge Zinnoxid enthielt, wahrscheinlich im Zustand des gewöhnlichen Zinnsteins.

Die Lösung der geglühten Steinmasse in Salzsäure wurde durch Schwefelsäure vom Baryt, und durch Schwefelwasserstoff-Schwefelammonium vom Nickeloxyd befreit, mit Salpetersäure oxydiert und abgeraucht, so dass die Schwefelsäure sowohl alle Salpetersäure als Salzsäure aus jagte, wieder in Wasser gelöst, mit ätzendem Ammoniak gefällt, darauf die in der Flüssigkeit zurückgeblieben Alkalien nebst dem Kalk und der Talkerde auf gewöhnliche Weise, d. h. mit essigsaurem Baryt, abgeschieden, was im Detail anzugeben mir überflüssig scheint.

Der Niederschlag mit ätzendem Ammoniak wurde wieder in Salzsäure gelöst und mit kohlensaurem Ammoniak gefällt; dabei blieb die Talkerde in der Flüssigkeit zurück, aus welcher sie gefällt und ihrem Gewichte nach bestimmt wurde.

Der Niederschlag gab mit ätzendem Kali Tonerde. Aus der Flüssigkeit, aus welcher die Tonerde mit kohlensaurem Ammoniak gefällt war, wurde durch Abdunsten und durch Glühen mit Salpeter eine deutliche Spur von Chrom erhalten, welche sich mit salpetersaurem Bleioxyd fällen liefs, aber doch nicht das Wägen verdiente.

Das mit ätzendem Kali behandelte Eisenoxyd aufs Neue gelöst und mit bernsteinsaurem Ammoniak gefällt, liefs etwas Nickeloxyd mit Talkerde in der Flüssigkeit zurück. Sie wurden auf vorhin genannte Weise geschieden.

Der Niederschlag mit bernsteinsaurem Ammoniak wurde geglüht und gewogen, dann zum feinsten Pulver gerieben und in einem Platintiegel mit Salpeter und etwas kohlensaurem Kali gebrannt. Die gelbe Salzmasse wurde mit Wasser ausgekocht, die Lösung mit Essigsäure gesättigt und gefällt zuweilen mit salpetersaurem Quecksilberoxydul, wo dann der geglühte Niederschlag Chromoxyd war, oder zuweilen mit salpetersaurem Bleioxyd, was chromsaures Bleioxyd gab. Aus beiden Niederschlägen wurde der entsprechende Gehalt an Chromeisen nach der Formel FeCr [1,3] berechnet. Da das gemeinschaftliche Gewicht des Eisen- und Chromoxyds bekannt war, wurde das des Eisenoxyds durch Abzug des vom Chromoxyd erhalten. Das Eisenoxyd wurde zu Oxydul berechnet, und davon die Menge abgezogen, welche nötig war, um mit dem Chromoxyd Chromeisen zu bilden.

Bei der Analyse des: löslichen wie des unlöslichen Teils der Bergart des Meteorsteins bestimmte ich den, Mangangehalt auf die Weise, dass ich alle erhaltene Talkerde sammelte, glühte und wog, darauf in Salzsäure löste, und zu dieser Lösung, nachdem sie in eine Flasche gegossen war, ein Gemenge von chlorigsaurem und doppelt kohlensaurem Natron groß; das Mangan fand sich dann nach 24 Stunden als Oxyd gefällt, welches nun geglüht, gewogen und von dem Gewichte der Talkerde abgezogen ward.

Das Glühen mit kohlensaurem Natron geht am besten, wenn man nicht beabsichtigt, das Chromeisen in Substanz abzuscheiden und die geringe Menge Alkali zu bestimmen. Auch bei dieser Methode bekommt man in dem mit bernsteinsaurem Ammoniak gefällten Eisenoxyd einen Hinterhalt von Chromoxyd.

Die Analyse durch Fluorwasserstoffsäure geht leicht. Nachdem die Säure im Wasserbade ab gedünstet ist, setzt man Schwefelsäure hinzu und vertreibt die Flusssäure aus dem Rückstand. Die Masse ist schwarz von unaufgelöstem Chromeisen, welches nach Verdünnung der Lösung mit Wasser und nach Digestion zur Auflösung des Gipses zurückbleibt. In Übrigen geschieht die Analyse ganz so wie bei der mit Baryt, nachdem dieser durch Schwefelsäure gefällt worden ist. Der Gehalt an Kieselerde gibt sich ‚durch den Verlust zu erkennen. Auch hier findet man, außer der Tonerde, eine Spur von Chrom vom ätzenden Kali ausgezogen, und in dem mit bernsteinsaurem Ammoniak gefällten Eisenoxyd gleichfalls eine Portion Chromoxyd.

Der Meteorstein von Blansko dieser Behandlung unterworfen, gab von dem durch Säuren zersetzbaren Minerale 51,5, von dem in Säuren unlöslichen 48,5. In einem anderen Versuch wurden 48,9 vom ersteren und 51,1 vom letzteren erhalten, woraus zu folgen scheint, dass das Gemenge nicht vollkommen homogen an allen Punkten ist.

Die Analyse des löslichen Minerals, auf 100 Teile berechnet, gab folgende Bestandteile:
\begin{center}
\begin{tabular}{ |l|r|r| }
    \hline
     & & Sauerstoff\\\hline
    Kieselerde & 33,084 & 17,192\\\hline
    Talkerde & 36,143 & 14,00\\\hline
    Eisenoxydul & 26,935 & 6,01\\\hline
    Manganoxydul & 0,465 & 0,12\\\hline
    Nickeloxyd, zinn- u. kupferhaltig & 0,465 & 0,10\\\hline
    Tonerde & 0,329 & 0,10\\\hline
    Natron & 0,857 & 0,12\\\hline
    Kali & 0,429 & 0,07\\\hline
    Verlust & 1,273 & \\\hline
     & 100,000 & \\
    \hline
\end{tabular}
\end{center}
\paragraph{}
Vergleicht man den Sauerstoffgehalt der Kieselerde mit dem der Basen, und erwägt dabei, dass Säuren bei Zersetzung des Minerals Schwefelwasserstoffgas entwickeln, so ergibt sich deutlich, dass hier eine Portion Eisen als oxydiert aufgeführt worden, die eigentlich geschwefelt war. Dadurch ist auch bei Zusammenrechnung der Resultate ein Verlust entsprungen, weil das Schwefelatom doppelt so schwer ist als das Sauerstoffatem. Es ist eine Unvollkommenheit bei der Untersuchung, dass die Menge des Schwefels nicht bestimmt wurde; allein dies würde die Genauigkeit der übrigen Bestimmungen verhindert haben, die ich doch für wichtiger hielt. Zwar habe ich gesucht bei einer anderen Portion den Schwefel zu bestimmen, aber doch keinen Gebrauch davon gemacht, weil sicher das Schwefeleisen ungleich verteilt ist, wovon man sich schon mit bloßem Auge überzeugen kann. Ich hatte mir eingebildet, dass man würde mit einem durch viel Wasser verdünnten Gemenge von Salzsäure und chlorsaurem Kali nur Nickeleisen und Schwefeleisen ausziehen und so das Mineral von diesen Verbindungen ganz befreien können, und unterwarf in dieser Absicht 3,5 Grm. vom gröblich zerstoßenen Steinpulver einer solchen Behandlung. Aus der Lösung konnte ich 0,13 Grm. schwefelsauren Baryt fällen, entsprechend 1/2 Proz. Schwefel von dem ganzen Stein, das Nickeleisen eingerechnet, oder 1,2 Proz. von dem löslichen Mineral; als ich indes das Ungelöste sehr gelinde erwärmte, wurde eine Portion Schwefel, welche die Säure unoxydiert abgeschieden hatte, teils sublimiert, teils in Brand gesetzt, so dass also der Schwefelgehalt grösser ist. Bei diesem Versuche fand ich auch, dass selbst die verdünnte Säure bedeutend von dem löslichen Minerale mit der Kieselerde und Allem auflöse; und bei Einziehung dieser Erfahrung hatte ich ungefähr die Hälfte von dem verloren, was ich zur Untersuchung anwenden konnte.

Durch diese Erörterung glaube ich es ganz wahrscheinlich gemacht zu haben, dass in dem in Säuren löslichen Meteormineral die Kieselerde und die Basen gleichviel Sauerstoff enthalten, und dass der Überschuss, den letztere davon enthalten, davon herrührt, dass das Schwefeleisen als Eisenoxydul berechnet wurde. Man könnte ihn auch davon ableiten, dass dem Steine Eisenoxyduloxyd eingemengt wäre, allein dessen Anwesenheit lässt sich nur in solchen Fällen ermitteln, wo es in bedeutenderer Menge vorkommt, wovon wir bei anderen Meteorsteinen Beispiele haben.

Das unlösliche Mineral wurde teils mit kohlensaurem Baryt und teils mit kohlensaurem Natron analysiert. Ich werde die Resultate beider Methoden anführen. Der Unterschied zwischen ihnen, ist wicht groß und kann in den Methoden begründet sein, aber auch in einer veränderlichen Mischung der Bestandteile des Minerals.
\begin{center}
\begin{tabular}{ |p{30mm}|p{18mm}|p{18mm}|p{24mm}| }
    \hline
      & Kohlensaur. Baryt. & Kohlensaur. Natron & Sauerstoffgehalt\\\hline
    Kieselerde & 57,145 & 57,012 & 29,626\\\hline
    Talkerde & 21,843 & 24,956 & 9,660\\\hline
    Kalk & 3,106 & 1,437 & 0,412\\\hline
    Eisenoxydul & 8,592 & 8,362 & 1,904\\\hline
    Manganoxydul & 0,724 & 0,557  & 0,124\\\hline
    Nickeloxyd, zinn- u. kupferh. & 0,021  & -,-  & -,-\\\hline
    Tonerde & 5,590 & 4,792 & 2,238\\\hline
    Natron & 0,931 & -,- & -,-\\\hline
    Kali & 0,010 & -,-  & -,-\\\hline
    Chromeisen (zinnhaltig) & 1,533 & 1,306 & -,-\\\hline
    Verlust & 0,505 & 1,579 & \\\hline
     & 100,000 & 100,000 & \\
    \hline
\end{tabular}
\end{center}
\paragraph{}
Der Verlust bei letzterer Analyse besteht hauptsächlich aus Alkali. Man sieht, der Sauerstoff der Kieselerde ist doppelt so groß als der der Basen. Legt man den Sauerstoffgehalt der Alkalien hinzu, so kommt der Sauerstoffgehalt der Basen neck näher an die richtige Zahl. Möglicherweise ist darunter eine geringe Portion eines Minerals enthalten, worin der Sauerstoffgehalt der Kieselerde das Dreifache des der Basen ist. Diele lässt sich nicht entscheiden, wenn man Gemenge analysieren muss.

Ich habe gesagt, dass sich in der Masse des Meteorsteins runde Kügelchen finden. Dies ist eine ganz gewöhnliche Erscheinung bei den Meteorsteinen. Schon Howard hat sie beobachtet und versucht sie zu analysieren. Ich konnte nicht so viel von ihnen abtrennen, um eine besondere Analyse mit denselben zu unternehmen, aber die Versuche, die ich mit ihnen anstellte, gaben ein dem Howard'schen gleiches Resultat, nämlich: dass sie dasselbe Mineral enthalten wie der Stein. Aus ihrem Pulver ließ sich nichts mit dem Magneten ausziehen, aber desungeachtet entwickelte es bei Übergießung mit Salzsäure den Geruch nach Schwefelwasserstoffgas. Ein Teil des Pulvers gelatinierte, ein anderer wurde nicht von der Säure verändert.

Ich führte vorhin an, die Analyse des Meteorsteins zerfalle in die Untersuchung des unmagnetischen und in die des magnetischen Teils. Ich komme jetzt zu der letzteren.

Um so viel wie möglich diesen Teil von dem Steinpulver zu befreien zerrieb ich ihn mit den Fingern unter Wasser, solange frisch aufgegossenes Wässer getrübt wurde. Dabei blieben Körner zurück, welche im Allgemeinen klein waren, aber vollkommen metallisch glänzten. Desungeachtet blieb doch in ihren Höhlungen sehr viel Steinmasse eingeschlossen, die erst bei der Auflösung frei ward und dadurch die Analyse verwickelter machte als geschehen müsste, weil die Bestandteile des löslichen Minerals sich mit dem Eisen des Meteors mischten, während das Ungelöste in Pulverform abgeschieden wurde. Der Gang der Analyse war folgender: 1,137 Grm. Meteoreisen wurde in Salzsäure gelöst und das Gas durch ein Gemenge von salpetersaurem Silberoxyd und Ammoniak geleitet. Es füllte sich Schwefelsilber, welches auf ein gewogenes Philtrum gebracht wurde; es wog 0,0215 Grm. = 0,0028 Grm. Schwefel. Nach beendeter Einwirkung der Salzsäure erschienen schwarze Punkte in dem Ungelösten, welche indes durch Zusatz von Salpetersäure und eine kurze Digestion verschwanden. Das ungelöste Mineral wog 0,1550 Gramm.

Die Lösung wurde mit Salpetersäure oxydiert, das Eisen mit bernsteinsaurem Ammoniak gefällt und nach dem Glühen gewogen. Bei Wiederauflösung in Salzsäure hinterliefs es etwas Kieselerde ungelöst; niedergeschlagen mit Schwefelwasserstoff-Schwefelammonium und die Flüssigkeit auf einen Gehalt von Phosphorsäure untersucht, erhielt ich eine geringe Spur von dieser, die indes, selbst in Form von phosphorsaurem Kalk, nicht gewogen werden konnte. Aus der mit bernsteinsaurem Ammoniak gefällten Flüssigkeit wurde das Nickeloxyd so nahe wie möglich mit Schwefelwasserstoff-Schwefelammonium abgeschieden, der Niederschlag geröstet, und das Nickeloxyd vom Kobaltoxyd nach der Phillips'schen Methode mit Ammoniak und Kali getrennt. Das Nickeloxyd wurde wieder in Salzsäure gelöst und mit Schwefelwasserstoff behandelt, wodurch ein erst gelblicher, nach dem Trocknen aber schwarzer Niederschlag entstand, worin sich mit dem Lötrohr Zinn und Kupfer entdecken lief.

Die mit bernsteinsaurem Ammoniak gefällte Flüssigkeit enthielt, nach Abscheidung der Metalle, noch Talkerde, Kalk und Kieselerde, welche auf gewöhnliche Weise voneinander getrennt wurden.

Die Analyse gab in den angeführten 1,137 Grm.
\begin{center}
\begin{tabular}{ l r }
    Eisenoxyd & 1,1940\\
    Nickeloxyd & 0,0555\\
    Zinnoxid und Kupferoxyd & 0,0050\\
    Kobaltoxyd & 0,0040\\
    Schwefel & 0,0028\\
    Kieselerde & 0,0275\\
    Talkerde & 0,0310\\
    Kalk & 0,0090\\
    Ungelöstes Mineral & 0,1550\\
     & 1,4830\\
\end{tabular}
\end{center}
\paragraph{}
Hievon muss, um dies Resultat anwendbar zu machen, das eingemengte Mineral abgerechnet werden. Ich setze dabei voraus, der Talkerdegehalt gebe nach der vorhin angeführten Analyse des auf löslichen Minerals an, wie viel Eisenoxyd diesem angehöre. Dies gibt 0,0252, entsprechend 0,023 Eisenoxydul. Daraus folgt, dass von den angewandten 1,137 müssen 0,2455 für beigemengtes Mineral abgezogen werden.

Von dem, was dann zurückbleibt, war 0,8115 Eisen, 0,0437 Nickel, 0,0030 Kobalt, 0,0040 Zinn und Kupfer, und 0,0028 Schwefel, welche, den 0,2455 hinzugefügt, 1,1105 ausmachen, und einen Verlust von 0,028 oder beinahe 2,5 Prozent ergeben. Letzterer entstand vermutlich aus eingesogener Feuchtigkeit, welche aus dem nicht sichtbaren und in unerwartet großer Menge dem Meteoreisen beigemengten Steinpulver nicht vollständig ausgetrieben worden war. Nach dieser Berechnung enthält das Meteoreisen:
\begin{center}
\begin{tabular}{ l r }
    Eisen & 93,816\\
    Nickel & 5,053\\
    Kobalt & 0,347\\
    Zinn und Kupfer & 0,460\\
    Schwefel & 0,324\\
    Spur vom Phosphor & \\
     & 100,000\\
\end{tabular}
\end{center}
\paragraph{}
Es ist, ah sich klar, dass diese Zahlen, die aus einer verwickelteren Analyse hergeleitet wurden als es der Fall gewesen sein würde, wenn kein Steinpulver mit gefolgt wäre, keine große Genauigkeit haben können. Wir werden weiterhin finden, dass das Meteoreisen Schwefel enthält, aber hier ist dessen Menge zu groß, als dass es hätte dem geschmeidigen Nickeleisen angehören können.

Zur Bestimmung des Zinn- und Phosphorgehalts bediente ich mich der zuvor, S. 10, erwähnten 3,5 Grm. Steinpulver, welche zur Bestimmung des Schwefelgehalts benutzt worden waren. Schwefelwasserstoff brachte eine gelbe Trübung hervor, worin sich viel überschüssiger Schwefel befand. Dieser hinterliefs nach dem Rösten einen Rückstand, welcher, mit Borax und kohlensaurem Natron reduziert, ein Zinnkorn gab. Darauf unvorbereitet und nur Schwefel in dem gelblichen Niederschlag erwartend, hatte ich den geglühten Rückstand vor Anstellung der Lötrohprobe nicht gewagt. Ich befürchtete nun Zinn in meinem destillierten Wasser. Ein Umstand, der keineswegs ungewöhnlich ist. Allein es fand sich, dass dasselbe nicht zu diesem Zinngehalt Anlass gegeben hatte. Ich leitete ihn nun von der Salzsäure her. Es geschieht nämlich oft, dass sich bei deren Bereitung in der Vorlage und dem Ableitungsrohr anfangs ein kristallinischer Anflug zeigt, welcher im Fortgang der Destillation verschwindet. Dieser flüchtige Stoff ist Zinnchlorid. Das Zinn fand sich dann zuvor in der Schwefelsäure. Meine Salzsäure, welche, mit vielem Wasser verdünnt, einem Strome Schwefelwasserstoff ausgesetzt wurde, ward endlich trübe, und setzte nach mehreren Tagen einen braunen Stoff in sehr geringer Menge ab, so dass er kaum zu einem Lötrohrversuch gesammelt werden konnte, aber in diesem war wirklich Zinn enthalten. Da dieser aus ungefähr 1/3 Pfund konzentrierter Salzsäure erhalten worden war, und bei meinen Versuchen nicht mehr als zwei bis drei Grammen auf einmal angewandt wurden, so ist klar, dass dieser Gehalt an Zinn und Kupfer wirklich aus dem Meteorstein herstammte, wovon ich mich auch dadurch vergewisserte, dass ich auch aus Chromeisen Zinn auszog. Inzwischen, da das Material noch nicht ganz verbraucht worden war, stellte ich für jeden Fall eine Gegenprobe an mit Salzsäure und Wasser, welche beide für sich mit Schwefelwasserstoff gesättigt und darauf von diesem Gase durch Erwärmung wieder befreit worden waren; allein das Resultat blieb in Bezug auf den Zinngehalt dasselbe.

Um mich des Phosphorgehalts in diesem Meteoreisen zu versichern, bediente ich mich wieder der eben erwähnten Lösung jener 3,5 Grm. Steinpulver, füllte daraus den Metallgehalt mit kohlensaurem Ammoniak‚ löste den Niederschlag in Salzsäure, versetzte die Lösung mit Schwefelwasserstoff-Schwefelammonium, filtrierte, nachdem sich das Schwefeleisen abgeschieden hatte, die gelbe Flüssigkeit ab, übersättigte sie mit Salzsäure, dunstete sie zur Trockne, löste den Rückstand wieder in Wasser, und vermischte die Lösung mit Chlorcalcium und ätzendem Ammoniak, wodurch ich 0,012 Grm. phosphorsauren Kalk erhielt.

In mineralogischer Hinsicht kann folglich der Meteorstein von Blansko angesehen werden als bestehend aus:

Nickeleisen, welches Kobalt, Zinn, Kupfer, Schwefel und Phosphor enthält | 17,15.

Silicat von Talkerde und Eisenoxydul, worin Basen und Kieselerde gleich viel Sauerstoff enthalten, nebst etwas Schwefeleisen | 42,67.

Silicat von Talkerde und Eisenoxydul, gemengt mit Silicaten von Alkali, Kalk und Tonerde, worin die Kieselerde doppelt so viel Sauerstoff als die Basen enthält | 39,43.

Chromeisen, verunreinigt mit Zinnstein | 0,75.

Dass die relativen Mengen dieser Gemengteil in verschiedenen Stücken des Steins Variationen unterworfen seien, darf kaum bezweifelt werden.

Ich habe der K. Akademie bereits zwei Mal Untersuchungen über Meteorsteine mitgeteilt. Die eine derselben hatte einen in Makedonien gefallenen Meteorstein zum Gegenstand.\footnote{Kongl. Vetensk. Acad. Handl. 1828. (Diese Annal. Bd. XVI, S. 611.)} Im Ansehen ist dieser bedeutend verschieden von dem von Blansko, aber das Resultat seiner Analyse gleicht dem eben angeführten so sehr, dass ich glaubte, diese Übereinstimmung auch bei andern nachsuchen zu müssen. Der Meteorstein aus Makedonien enthält Meteoreisen, worin sich 6 Prozent kobalthaltiges Nickel und viel Schwefeleisen fand, und der unmagnetische Teil war zerlegbar in 47,5 Prozent eines löslichen und 52,5 Prozent eines unlöslichen Minerals. In dem löslichen enthielten die. Basen mehr als gleichviel bis anderthalb Mal so viel Sauerstoff wie die Kieselerde. Es ist aber wahrscheinlicher, dass dies von eingemengtem Schwefeleisen und Magneteisenstein herrührt, als von der Gegenwart eines so basischen Silicats. Das Unlösliche bestand aus Silicaten von Talkerde, Eisenoxydul, Kalk, Alkali und Tonerde, worin die Kieselerde zwei Mal so viel Sauerstoff als die Basen enthielt.\footnote{In der Abhandlung steht, dass der Sauerstoff gleich sei. Diess ist aber ein Versehen bei der Abfassung; denn der Sauerstoff in der Kieselerde ist 13,6 und der in sämtlichen Basen 6,5.} Die andere Analyse wurde mit einem in Böhmen gefundenen Meteoreisen angestellt,\footnote{Kongl. Vetensk. Acad. Handl. 1832, p. 106. (Diese Annalen, Bd. XXVII, S. 118.)} worin enthalten waren: Eisen 92,473, Nickel 5,667, Kobalt 0,235 mit Spur von Schwefel und eine unlösliche Verbindung von Phosphor mit Eisen und Nickel.

Die Frage war also ganz natürlich: Sind alle Meteorsteine Gemenge von Nickeleisen und Schwefeleisen mit in Säuren löslichen Silicaten von Talkerde und Eisenoxydul, und in Säuren unlöslichen Silicaten von Talkerde, Tonerde und Alkali, nebst Chromeisen und Zinnstein; sind ferner die Meteorsteine immer zinnhaltig, immer gemengt, mit Phosphorverbindungen?

Die Analysen meiner Vorgänger beantworten diese Fragen nicht. Howard hatte wohl Nickeleisen und Schwefeleisen aus der magnetischen Bergart ausgeschieden, welche er, wie Alle nach ihm, als eine einzige Verbindung analysierte; allein weiter war er nicht gegangen. Laugier hatte den Chromgehalt entdeckt, über welchen Stromeyer die Vermutung äußerte, dass er von eingemengtem Chromeisen herrühre, wie es durch die obigen Versuche bewiesen worden ist. Dies veranlasste die Ausführlichkeit der gegenwärtigen Untersuchung, wobei ich zum Gegenstande meiner Untersuchung solche Meteorsteine wählte, welche in ihrem Ansehen sehr von den gewöhnlichen abweichen, dabei annehmend, dass die, welche einander vollkommen gleichen, ohne Irrtum als von demselben Orte abstammend und als gleiche Zusammensetzung besitzend angesehen werden können. Ich habe mich jedoch hierbei auf die wenigen beschränken müssen, die in meiner eigenen Sammlung befindlich sind.

\section{Meteorstein von Chantonnay.}
\paragraph{}
Dieser fiel unter den gewöhnlichen Erscheinungen einer Feuerkugel und unter einem donnerähnlichen Getöse um 2 Uhr Morgens am 5. Aug. 1812 nicht weit von Chantonnay im Departement de la Vendée, und ward an demselben Tage von dem Pächter des Guts la haute Révetison auf einem Felde in der Nähe seines Wohnhauses aufgefunden. Er war dritte halb Fuß tief in die Erde eingedrungen und roch noch stark nach Schwefel. Er wog 69 Pfund, und besaß eine viel größere Härte und Kohäsion als gewöhnlich die Meteorsteine, so dass er am Stahle Funken gab. Seine Bruchfläche hatte eine dunklere Farbe als gewöhnlich die Meteorsteine, und an einigen Stellen war sie ganz schwarz. Die umgebende verglaste Rinde war weniger schwarz und zuweilen dunkel graurot. Ich weiß nicht, dass eine Analyse desselben bekannt gemacht worden sei. Das Stück, welches ich davon besitze, habe ich von den verstorbenen französischen Mineralogen Lucas erhalten, und seine Kennzeichen stimmen ganz mit der Beschreibung überein, welche kurz nach dem Falle dieses Steins von Chladni während seines Aufenthaltes in Paris gegeben wurde.\footnote{Gilberts Annalen, Bd. LX, S. 247.}

Zur Analyse habe ich nur das Schwärzeste und Härteste, im Ansehen von den gewöhnlichen Meteorsteinen ganz Verschiedene Angewandt. Diess enthält Nickeleigen in größeren und kleineren Körnern, und viel Schwefeleisen, welche beide mit dem Magneten ausgezogen werden können. Diese habe ich, als meinem Zweck nicht angemessen, nicht analysiert. Ich hatte zur Absicht, Schwefeleisen daraus zur Untersuchung zu erhalten; allein unter dem Mikroskope entdeckte ich darin bald zahlreiche Flitterchen von Nickeleisen und abgeschiedene Teile vom Steinpulver, welche beim Ausziehen mit dem Magneten daran hängen geblieben waren. Das Steinpulver wurde mit dem Magneten unter Wasser behandelt, und wiewohl es mir schien, als sei es ganz frei von magnetischen Teilen, so wurde doch, bei Übergießung mit Salzsäure, Schwefelwasserstoffgas daraus entwickelt.

Der von Säuren zersetzbare Teil machte 51,12 Prozent, und der in ihnen unlösliche 48,88 Prozent aus, also gerade so viel wie beim Stein von Blansko.

Das lösliche Mineral enthielt:
\begin{center}
\begin{tabular}{ |p{30mm}|r|r| }
    \hline
     &  & Sauerstoffgeh.\\\hline
    Kieselerde & 32,607 & 16,96\\\hline
    Talkerde & 34,357 & 13,29\\\hline
    Eisenoxydul & 28,801 & 6,59\\\hline
    Manganoxydl & 0,821 & 0,19\\\hline
    Nickeloxyd, verunreinigt mit Zinn- und Kupferoxyd & 0,456 & \\\hline
    Natron und Kali & 0,977 & \\\hline
    Verlust & 1,971 & \\\hline
    & 100,000 & \\
    \hline
\end{tabular}
\end{center}
\paragraph{}
Im unlöslichen Minerale fand sich:
\begin{center}
\begin{tabular}{ |p{30mm}|r|r| }
    \hline
    Kieselerde & 56,252 & 29,75\\\hline
    Taikerde & 20,396 & 7,91\\\hline
    Kalk & 3,106 & 0,88\\\hline
    Eisenoxydul & 9,723 & 2,21\\\hline
    Manganoxydul & 0,690 & 0,16\\\hline
    Nickeloxyd mit Zinn- u. Kupferoxyd & 0,138 & 0,05\\\hline
    Tonerde & 6,025 & 2,81\\\hline
    Natron & 1,000 & 0,26\\\hline
    Kali & 0,512 & 0,08\\\hline
    Chromeisen & 1,100 & \\\hline
    Verlust & 1,070 & \\\hline
     & 100,000 & \\
    \hline
\end{tabular}
\end{center}
\paragraph{}
Hier findet sich also, ungeachtet der Ungleichheit im Ansehen, eine wunderbare Gleichheit in der Zusammensetzung. Beim Vergleiche zeigt sich, dass einem größeren Kalkgehalt ein größerer Tonerdegehalt folgt. Daraus lässt sich schließen, dass diese Stoffe Bestandteile eines besonderen. Minerals ausmachen, welches in verschiedener Quantität eingemengt sein kann.

Da mir von diesem Meteorsteine mehr zu Gebote stand als von anderen, so suchte ich mir einen Begriff von dem Zinngehalt darin zu verschaffen, und zwar durch folgende Versuche: 2,93 Grm. des geschlemmten und von den magnetischen Teilen befreiten Steinpulvers wurden durch Fluorwasserstoffsäure zerlegt, was mit vieler Heftigkeit geschah. Nach Abdunstung wurde die Fluorwasserstoffsäure mit Schwefelsäure ausgetrieben und die Salzmasse in Wasser gelöst, wobei 0,025 Grm. Chromeisen ungelöst blieben. Durch Schwefelwasserstoff wurde aus der Lösung Schwefelzinn gefällt, welches nach dem Rösten 0,002 Grm. wog, und bei Reduktion ein, wegen beigemengten Kupfers, ins Gelbe fallende Korn gab. Das Chromeisen wurde durch Schmelzen in saurem schwefelsauren Kali gelöst, und die Lösung mit Schwefelwasserstoffgas behandelt, was noch 0,0015 Grm. Zinnoxid gab, worin vor dem Lötrohr eine sehr schwache Spur Kupfer entdeckt werden konnte. Dieser Meteorstein enthält also ungefähr 1/10 Prozent Zinnoxid und 0,84 Prozent Chromeisen. Daraus folgt dann, dass der bei der vorhergehenden Analyse in dem Unlöslichen gefundene Chromeisengehalt zu niedrig ist. Dies rührte davon her, dass das mit bernsteinsaurem Ammoniak gefällte Eisenoxyd, nach dem Glühen und Wägen, verlor ehe es mit Alkali und Salpeter geschmolzen wurde, welches aber für nicht so wichtig gehalten wurde, dass es einen neuen Versuch verdient hätte. Nach dem nun Angeführten hätte das Chromeisen 1,7 Prozent vom unlöslichen Minerale betragen müssen.

\section{Meteorstein von Lontalax.}
\paragraph{}
Dieser Meteorstein fiel am 13. Dez. 1813 in der Nähe des Dorfes Lontalax, im Kirchspiel Savitaipals im Län Viborg in Finnland, Ein großer Teil der Stücke fiel auf das Eis, von wo sie aufgelesen wurden.\footnote{A. N, Scheerer, Allgemeine nordische Annalen, Bd. I, S. 174.} Er ist von Nordenskiöld\footnote{Bidrag til närmare kännedom of finlands mineralier och geognosie, I p. 99.} näher beschrieben, welcher die Güte hatte mir ein kleines Stück davon mitzuteilen, von dem ich den größten Teil zu der folgenden Untersuchung anwandte.

Nach Nordenskiölds Angabe enthält dieser Meteorstein als Gemengteil folgende: 1. Ein hellolivengrünes Mineral, welches sich vor dem Lötrohr wie Olivin verhält, nur in geringer Menge vorkommt und nicht grösser ist als ein Stecknadelknopf. 2. Ein halb klares, weißes, blättriges Mineral, welches auf der Oberfläche kristallinisch aussieht und leicht zerbröckelt. 3. Schwarze, dem Magnete folgsame Punkte. 4. Ein aschgrauer, wenig zusammenhängender Stoff, welcher ohne Aufschwellen zu einer schwarzen Kugel schmilzt und die reichlichste Masse des Steins ausmacht. Auswendig ist er von einer schwarzen Schlackenrinde umgeben.

Was ich von diesem Meteorstein erhielt, bestand fast nur aus dem unter No. 2 angeführten Teil, gemengt mit einigen schwarzen Punkten; die aschgraue Hauptmasse fehlte aber gänzlich.

Ich werde das von mir zur Analyse angewandte Stück kurz beschreiben. Es ist im Vergleich mit gewöhnlichen Meteorsteinen weiß, neben weißen Mineralien aber graulich, kaum werklich ins Grüne fallend. Hie und da sind schwarze Punkte eingesprengt, welche dem Magneten folgen und sich in Salzsäure, ohne Geruch nach Schwefelwasserstoffgas und ohne Gasentwicklung, zu einer dunkelgelben Flüssigkeit auflösen, woraus also folgt, dass sie aus Eisenoxyduloxyd oder Magneteisenstein bestehen. Es ist übrigens ein Aggregat von Teilen, welche, ohne gerade kristallisiert zu sein, doch kristallinisches Gefüge haben, und so locker zusammenhängen, dass der Stein sich mit Leichtigkeit zerbrechen lässt. Die Brocken, die dabei abfallen, gleichen sehr dem zarten Pulver von glasigem Feldspat, was Nordenskiöld auf die Vermutung brachte, sie seien Leucit. - Sein Pulver ist rein weiß. Vor dem Lötrohr wird es augenblicklich schwarz, und nach dem Erkalten dunkelrot, Im Übrigen verhalt es sich vor dem Lötrohr ganz wie der Stein von Blansko.

1,22 Grm. des fein geschlemmten Pulvers, aus dem Alles dem Magneten Folgsame vor dem Zerreiben ausgezogen, und welches vor dem Wägen bei +150° C. getrocknet worden war, hinterliefs nach Behandlung erst mit Königswasser und dann mit kohlensaurem Natron 0,07875 Grm. ungelöst. Das Resultat fiel folgendermaßen aus:
\begin{center}
\begin{tabular}{ |p{30mm}|r|p{20mm}|p{24mm}| }
    \hline
     & Ganze Masse & Das Lösliche in Prozenten & Sauerstoffgehalt\\\hline
    Kieselerde & 0,425 & 37,411 & 19,44\\\hline
    Talkerde & 0,344 & 32,922 & 12,74\\\hline
    Eisenoxydul & 0,325 & 28,610 & 6,51\\\hline
    Manganoxydul & 0,009 & 0,793 & 0,17\\\hline
    Tonerde & 0,003 & 0,264 & 0,12\\\hline
    Kupferoxyd, Zinnoxid, Kali und Natron\footnote{Der Gehalt an Zinnoxid war ungefähr der gewöhnliche der Meteorsteine; allein der Kupferoxydgehalt war so gering, dass sich die Reaktion desselben schwer vor dem Lötrohr hervorbringen ließ.} & Spur & Spur & \\\hline
    Unlösliches & 0,079 & & \\\hline
     & 1,215 & 100,000 & \\
    \hline
\end{tabular}
\end{center}
\paragraph{}
Der Zufall hat mich also zu einer Probe von dem Mineral geführt, welches die Hauptmasse des in Säuren löslichen Bestandteils der Meteorsteine ausmacht, woraus der Schluss gezogen werden kann, dass dieser Bestandteil ein Silicat von Talkerde und Eisenoxydul ist, wahrscheinlich in variierenden gegenseitigen Verhältnissen, aber in welchem die Kieselerde eben so viel Sauerstoff als die Basen enthält. Der Überschuss in dem letzteren, welcher sich bei der vorgehenden Analyse zeigt, rührt offenbar zum Teil von eingemengtem Schwefeleisen her, welches bei der Analyse oxydiert erhalten wurde; ob aber dabei zugleich Eisenoxyduloxyd oder ein basischeres Silicat vorkommt, lassen meine Versuche unentschieden.

Das hier analysierte Mineral gibt ziemlich ungezwungen die Formel F(S + 2MS); inzwischen hat man Grund zu vermuten, dass das Atomverhältnis ein zufälliges sei, und dass der Meteor-Olivin diese isomorphen Silicate in variierenden Verhältnissen enthalte.

Derjenige Teil des Steins von Lontalax, welcher sich nicht in Säure und kohlensaurem Natron löst und 6,37 Prozent vom Gewicht des Steins’ ausmacht, hinterließe nach Behandlung mit Fluorwasserstoffsäure ungefähr 1 Proz. (0,0127. des analysierten Quantums) Chromeisen ungelöst, dessen Verhalten vor dem Lötrohr die Gegenwart von Zinnoxid dartat. Die, Flusssäure hatte aufgelöst: Talkerde, Kalk, Eisenoxydul, Tonerde und Manganoxydul, in einen Verhältnisse, welches zu zeigen schien, dass dieses Unlösliche gleiche Zusammensetzung habe, wie das unlösliche Mineral in den vorhergebenden Meteorsteinen.

\section{Meteorstein von Alais.}
\paragraph{}
De Fall dieses Meteorsteins ereignete sich am 15. März 1806 um 5 1/2 Uhr Nachmittags in der Nachbarschaft von Alais in Frankreich. Es wurden zwei Knalle gehört und es fielen zwei Steine nieder, der eine bei St. Etienne de Lolm und der andere bei Valence, beides Dörfer, das erstere 4 1/2, das letztere 2 Lieues von Alais entfernt. Bei Valence schlug der fallende Stein einen Ast von einem Feigenbaum. An beiden Orten wurde der Fall von glaubwürdigen Personen bezeugt, welche die Steine auflasen. Der erste wog acht, der letztere ungefähr vier Pfund. Sie zersprangen beim Fall. Dieser Stein ist von allen andern verschieden. Er gleicht einem verhärteten Ton und zerfällt in Wasser mit Tongeruch. Thénard, welcher ihn zuerst untersuchte, fand darin, außer den gewöhnlichen Bestandteilen der Meteorsteine, eine Portion Kohle, welche Angabe später auch Vauquelin bestätigte.\footnote{Gilb. Annal. der Physik, Bd. XXIV, S. 193.}

Durch den französischen Mineralogen Lucas habe ich eine ganz geringe Probe von diesem Meteorstein erhalten, dieselbe aber immer für einen Brocken der Akkererde angesehen, auf welche der Stein herabfiel. In diesem Argwohn wurde ich bestärkt durch das Verhalten der Masse zum Wasser bei der Vorbereitung zur Analyse, und ich war nahe daran, sie ganz fortzuwerfen. Ehe ich aber dazu schritt, las ich die Urkunden darüber nach, und fand deren Angaben so übereinstimmend mit dem, was ich vor mir hatte, dass ich mit umso größerem Interesse die Untersuchung fortsetzte. Es entstand nämlich die Frage: Enthält diese kohlenhaltige Erde wohl Humus oder eine Spur von anderen organischen Verbindungen? Gibt dies möglicherweise einen Wink über die Gegenwart organischer Gebilde auf anderen Weltkörpern?

Ich will hier zunächst eine kurze Beschreibung des Steines geben, nach dem Stück, welches ich davon besitze. Die Farbe ist schwarz, etwas ins Graue fallend, mit dichten, feinen, weißen Punkten oder einem Anflug. Dies findet sich nicht in den älteren Beschreibungen angegeben; allein im Dictionnaire des sciences naturelles, XXX p. 339, heißt es, dass dieser Meteorstein die Neigung habe, sich mit einer Effloreszenz zu bekleiden, welche die Verfasser für Eisenvitriol ausgeben. Der Stein ist leicht zu zerbrechen, und zerbröckelt schon zwischen den Fingern. Gerieben mit dem Nagel oder einem andern glatten Körper nimmt er Politur an, wie es oft mit Tonarten der Fall ist. In Wasser gelegt zerfällt er nach einigen Augenblicken zu einem graugrünen Brei von einem starken Tongeruch, mit einem nicht unangenehmen Nebengeruch nach frischem Heu. Geschlemmt und sodann getrocknet hat das Pulver eine aus Schwarz, Grün und Braun zusammengesetzte Farbe. Vor dem Lötrohr in einem Kolben erhitzt, gibt es Wasser, schweflige Säure und endlich ein dunkelbraunes Sublimat, aber kein brenzliches Oel. Der Rückstand ist rußschwarz und lässt sich an offner Luft rot brennen. Er schmilzt äußerst träge zu einer schwarzen schlackigen, nicht gehörig geflossenen Masse. Mit Flusssäure verhält er sich ganz wie gewöhnliche Meteorsteine Der Magnet zieht aus ihm eine schwarze, nicht glänzende Masse, welche sich sehr schwer von dem tonartigen Muttergestein befreien lässt.

Das Wasser, worin der Meteorstein zerfallen ist, enthält ein aus dem Stein gezogenes Salz. 89,7 Th. des geschlemmten und bei 100° C. getrockneten Steinpulvers entsprechen 10,3 Th. Salz, im wasserfreien Zustand gewogen. Aus diesen 89,7 Th. zog der Magnet 11,92 Th. aus; allein unter diesen befand sich noch viel Muttergestein, welches ich nicht abzuscheiden vermochte.

a. Das mit dem Magneten Ausgezogene enthielt feine, weiße, metallisch glänzende Flitterchen, aber in geringer Menge und größtenteils nur unter dem Mikroskop erkennbar. Diese Flitterchen, ausgesucht und in Salzsäure gelegt, lösten sich unter Entwicklung von Wasserstoffgas auf, waren also metallisches Eisen. Um zu bestimmen, ob sich auch Nickel darin befände, hatte ich zu wenig von den Metallflitterchen; allein ich zweifle an dessen Gegenwart, da sich das Nickel oxydiert in dem Steinpulver befindet. Das Übrige des Magnetischen löste sich in Salzsäure, ohne Gasentwicklung mit dunkelgelber Farbe, und schwachem, aber unzweideutigem Geruch nach Schwefelwasserstoff. Das Magnetische bestand also aus ganz Wenig metallischen Eisens, etwas Schwefeleisen und meistens aus Eisenoxyd-Oxydul. Eine Spur von Chrom konnte ich durch Schmelzen mit etwas Alkali und Salpeter nicht darin entdecken.

ß. Das mit Wasser Ausgezogene gab eine blass gelbliche, im Allgemeinen schwach gefärbte Lösung, welche nach freiwilliger Abdunstung eine kristallisierte, nicht verwitternde Salzmasse hinterliefs. Ein Teil dieser Salzmasse wurde zur Verjagung des Krystallwassers erhitzt. Bei einer Temperatur, welche noch nicht bis zum Glühen ging, würde sie braun gebrannt unter einem brenzlichen Geruch; dann in Wasser gelöst, setzte sie einen schwarzbraunen kohligen Stoff ab, welcher, getrocknet, ohne Rückstand verbrannte. Das Wasser hatte also einen organischen Stoff ausgezogen, der im Vergleich mit dem Kohlengehalt, den er hinterliefs, oder mit der dunkeln Farbe, die er beim Erhitzen annahm, wenig gefärbt war. Ungeachtet des großen Interesses, welches die nähere Kenntnis der Eigenschaften dieses Stoffes besaß, musste ich mich damit begnügen, seine Anwesenheit erkannt zu haben. Wenn das Steinpulver gut mit warmem Wasser ausgezogen war, löste weder Ammoniak noch ätzendes Kali einen organischen Stoff mehr von ihm auf.

Ein Teil des kristallisierten Salzes, welches im lufttrocknen Zustand 0,285 Grm. wog, wurde in Wasser gelöst und mit einem Paar Tropfen kohlensauren Ammoniaks versetzt. Es entstand dadurch kein Niederschlag, zum Beweise, dass das Salz kein Eisen enthielt, also nicht Eisenvitriol war. Etwas Schwefelwasserstoff-Schwefelammonium gab einen schwarzen Niederschlag, welchen ich in einer verkorkten Flasche sich absetzen ließ. Dieser gab 0,005 Grm. Nickeloxyd, welches sich vor dem Lötrohr als verunreinigt mit Kupfer erwiess, mit Phosphorsalz aber nicht die Opalisierung beim Erkalten gab, welche die Gegenwart des Zinnoxyds anzeigt. Das Nickeloxyd entspricht 0,01 schwefelsauren Nickeloxyds.

Durch Zersetzung mit essigsaurem Baryt und andere gewöhnliche analytische Methoden wurden daraus erhalten: 0,04 Grm. Talkerde, entsprechend 0,118 Grm. wasserfreier schwefelsaurer Talkerde, 0,034 Grm. schwefelsauren Natrons, 0,004 Grm. schwefelsauren Kalis und 0,012 Grm. schwefelsauren Kalks erhalten, oder zusammen 0,178 Salze und 0,107 Krystallwasser, das vermutlich nicht auf jedes einzelne Salz, sondern auf, die Doppelsalze aus Natron und Kali mit Talkerde und Nickeloxyd und auf eine Portion freier schwefelsaurer Talkerde verteilt werden muss.

In diesem Salze findet sich überdies noch eine Spur von schwefelsaurem Ammoniak. Mengt man den Stein mit Wasser und lässt ihn darin zerfallen, so entsteht ein sehr starker Heugeruch, und ein mit Salpetersäure benetzter Glaspfropfen darüber gehalten, gibt weiße Nebel, zwar in geringer Menge, aber ganz sichtbar. Dieser Ammoniakgehalt ist jedoch wahrscheinlich nicht ursprünglich, denn wenn man das Steinpulver mit Ammoniak behandelt, gut mit Wasser auslaugt, und, nach dem Trocknen im Wasserbade, einer trocknen Destillation aussetzt, so erhält man ein stark ammoniakhaltiges Wasser, was nicht der Fall ist bei Behandlung mit Wasser. Dieses Ammoniak kann also sehr wohl während der 28jährigen Aufbewahrung im Mineralienschrank hineingekommen sein.

Von Wichtigkeit wäre es gewesen, sogleich nach dem Fall des Steins zu ermitteln, ob derselbe dieses Salz fertig gebildet, und letzteres dann Krystallwasser enthalten habe, wodurch die Frage: ob Wasser in der Heimat der Meteorsteine vorhanden sei, beantwortet werden könnte. Nun lässt sich zwar vermuten, dass das Salz aus einem Talkerdesilikat und Schwefeleisen entstanden sei, indem sich letzteres in Eisenvitriol verwandelte, dieses von der Talkerde zersetzt wurde und das Eisenoxydul in Eisenoxyduloxyd überging. Erwägt man indes, dass Thénard angibt, einerseits, dass der Stein mit Säuren sehr wenig Schwefelwasserstoff entwickelte, und andererseits, dass er, nach Verpuffung mit Salpeter, einen Niederschlag mit Chlorbarium, der 3 1/2 Proz. Schwefel entsprach, lieferte, so muss man schließen, dass der Stein entweder gewöhnlichen Schwefelkies oder ein schwefelsaures Salz enthielt. Beide Fälle sind sicher ungewöhnlich, aber der letztere zeigte sich in meinen Versuchen als wirklich vorhanden, und wahrscheinlich fand er auch statt, als Thénard seine Versuche anstellte, denn er erhielt 17 Prozent Wasser, was weit mehr ist als der vom Salz befreite Stein enthält, und zeigen würde, dass der Stein vom Anfang an wasserhaltig war. Indes stellte Thénard die Analyse ungefähr zwei Monate nach dem Fall des Steines an, und in der Zeit konnte das fein zerteilte Salz eine große Portion Krystallwasser aufgenommen haben. Es ist nämlich möglich, dass der Anflug von Bittersalz sich allmälig in der Atmosphäre der Erde gebildet hatte, dadurch, dass das Salz Krystallwasser aufnahm und durch den Zutritt des Wassers aufschwoll.

$\gamma$. Das mit Wasser ausgelaugte Steinpulver enthält, nach Thénards Analyse, eine Portion Kohle. Es war natürlich zu vermuten, dass diese Kohle mit Wasserstoff und Sauerstoff, vielleicht auch mit Stickstoff eine Verbindung ausmachte. Da weder Kali noch Ammoniak eine organische Verbindung auszog, so blieb nur übrig, die Produkte der trocknen Zersetzung zu studieren, weil die Lösung dieser Verbindung in Säuren, sie mit den übrigen unorganischen Stoffen, die sich entweder gelöst hätten oder um gelöst geblieben wären, vermengt haben würde. Das wohl ausgekochte, geschlemmte und bei 100° C. getrocknete Pulver wurde demnach in einem kleinen Destillationsapparat bis zum Glühen erhitzt, und das sich entwickelnde Gas in eine umgestürzte, mit Kalkwasser gefüllte Flasche geleitet. 1,586 Grm, Steinpulver hinterließen 1398 Grm. kohlschwarzen Rückstands.

Keine Tropfen von brenzlichem Oel zeigten sich im Retortenhalse; dagegen sammelte sich viel und ungefärbtes Wasser. Ein schmaler Streifen Lackmuspapier, in den Hals der Retorte gesteckt, ward rot. Das Gas wurde unter starker Trübung vom Kalkwasser absorbiert; und es blieb sehr wenig unverschluckt, nicht mehr als die Luft des Gefäßes betrug, und darin schien kein fremdes Gas enthalten zu sein. Thénard erhielt brennbare Gase, aber bei seinem Versuch war der in Wasser lösliche Stoff nicht entfernt worden. Das Kohlensäuregas gab 0,15 kohlensauren Kalk, entsprechend 0,0696 Grm. Kohlensäure oder 0,01813 Grm. Kohle.

Das Wasser im Retortenhalse besaß einen starken Geruch nach Heu oder richtiger nach Tonkabohnen, welcher beim Trocknen verschwand. Im hinteren Teil des Retortenhalses fand sich eine geringe Spur eines weißen Salzes, nebst einer Portion eines schwarzbraunen Sublimats. Diess weiße Salz war löslich im Wasser und ätzendes Kali entwickelte eine Spur von Ammoniak daraus, aber die Lösung ward nicht durch salpetersaures Silberoxyd gefällt. Ich konnte nicht ermitteln, mit welcher Säure das Ammoniak verbunden war.

Dieses braune Sublimat ist ein mir gänzlich unbekannter Körper. Er machte von der angewandten Quantität 0,015 Grm. aus. Dieser geringen Menge wegen konnte ich von seinen Eigenschaften nur folgende ermitteln. Seine Farbe ist in dünnen Kanten beim Hindurchsehen schwarzbraun, beim Darauf sehen fast schwarz. Die der Röhre zugewandte Seite ist dunkelgrau und etwas glänzend. Er hat weder Geschmack noch Geruch, wenn der Heugeruch verschwunden ist. In sauerstofffreier Luft kann er sublimiert werden; ohne Anzeigen von Kristallisation. In gewöhnlicher Luft oder in Sauerstoffgas verwandelt er sich in einen weißen Rauch, welcher sich an kalte Körper anlegt. Dieser Rauch hat einen stechenden Geruch. Der weiße Anflug ist löslich in Wasser, reagiert nicht auf Lackmuspapier und wird van salpetersaurem Silber nicht gefällt. Wenn er in Sauerstoffgas verbrennt wird, zeigt sich keine Spur von Feuchtigkeit; Kalkwasser wird nicht von dem Gase getrübt und es setzt sich nach einer Weile keine Spur von kohlensaurem Kalk ab. Der braune Körper ist unlöslich in Wasser, Ammoniak, ätzendem Kali, Salzsäure, kochender Salpetersäure von 1,24 spez. Gew. War er ein Produkt der trocknen Destillation oder fand er sich fertig gebildet vor und wurde durch die Hitze sublimiert? Ich kann diese Fragen nicht genügend beantworten.\footnote{Wollte man fragen: War er ein einfacher brennbarer Körper, so würde man vielleicht zu viel Gewicht darauflegen.}

Aus dem nun Angeführten folgt, dass die bei 100° C. getrocknete, von löslichen. Substanzen befreite Meteormasse gegeben hat:
\begin{center}
\begin{tabular}{ l r }
    Schwarzen geglühten Rückstand & 88,146\\
    Graubraunes Sublimat & 0,944\\
    Kohlensäuregas & 4,328\\
    Wasser & 6,582\\
    & 100,00\\
\end{tabular}
\end{center}
\paragraph{}
Analyse des schwarzen geglühten Rückstands. Er wog 1,382 und wurde mit Salzsäure zersetzt. Die Lösung war sehr dunkelgelb, das Ungelöste schwarz. Schwefelwasserstoff nahm die Farbe der Lösung fort. Der dabei erhaltene Niederschlag hinterließ, nach dem Fortbrennen des Schwefels, 0,005 Grm. Zinnoxid, verunreinigt mit Kupferoxyd. Übrigens wurde die Analyse ganz nach dem bereits mitgeteilten allgemeinen Plan angestellt. Die Lösung der Kieselerde in kohlensaurem Natron war gelblich, ward aber bei Sättigung mit Säure farblos. Nach Abscheidung der Kieselerde durch Abdunstung, wurde, aus der Auflösung des Salzes in Wasser, mit einem Tropfen ätzenden Ammoniaks 0,006 Grm. Zinnoxid gefällt. Der nach Behandlung mit kohlensaurem Natron unlösliche Teil war kohlschwarz. Ein Versuch, die Kohle in Sauerstoffgas fortzubrennen und die Kohlensäure aufzufangen glückte nur teilweise, weil sie sich nur an der Oberfläche oxydierte und rot ward. Sie gab dabei Wasser ab. Das Geglühte wog 0,12. Die angewandten 1,382 hatten also gegeben:
\begin{center}
\begin{tabular}{ |p{35mm}|p{20mm}|p{24mm}| }
    \hline
     &  & Sauerstoffgehalt\\\hline
    Kieselerde & 0,4315 & 21,5\\\hline
    Talkerde & 0,3070 & 11,88\\\hline
    Kalk & 0,0032 & 0,09\\\hline
    Eisenoxydul & 0,4011 & 9,13\\\hline
    Nickeloxyd & 0,0190 & 0,40\\\hline
    Manganoxydul & 0,0036 & 0,07\\\hline
    Tonerde & 0,0325 & 1,52\\\hline
    Chromeisen, zerlegt & 0,0087 & \\\hline
    Zinnoxid, kupferhaltig & 0,0110 & \\\hline
    Unlöslicher kohlenhaltig. Rückstand & 0,1200 & \\\hline
    Verlust & 0,0640 & \\\hline
     & 1,3820 & \\
    \hline
\end{tabular}
\end{center}
\paragraph{}
Der Verlust, welcher ungefähr 4 Prozent beträgt, ist etwas groß. Ein Teil davon ist Sauerstoff im Eisenoxyduloxyd. Übrigens stellt sich hier zwischen dem Sauerstoff der Kieselerde und dem der Basen dasselbe Verhältnis ein wie bei den vorhergehenden Meteorsteinen. Der Überschuss in dem letzteren hat vermutlich hier dieselbe Ursache wie dort, und rührt hier noch deutlicher von eingemengtem Eisenoxyduloxyd her.

Der unlösliche kohlschwarze Teil dieses Meteorsteins wurde erst mit Fluorwasserstoffsäure und dann mit Schwefelsäure behandelt, worauf ein schwarzes Pulver ungelöst blieb. Dieses wurde auf ein gewogenes Philtrum gebracht und darauf eine gewogene Portion desselben im Sauerstoffgas verbrannt, das, in einer Liebig'schen Absorptionsröhre, über dieselbe und dann in ein Gemenge von ätzendem Ammoniak und Chlorcalcium geleitet ward.

Die ammoniakalische Flüssigkeit hatte 48 Stunden lang in einer verkorkten Flasche gestanden, ehe sie in die Absorptionsröhre gefüllt worden war, so dass also aller kohlensaurer Kalk, der durch einen Kohlensäuregehalt des Ammoniaks gebildet worden sein könnte, sich abgesetzt haben musste. Sie wurde 24 Stunden lang verschlossen stehen gelassen, wo dann die klare Flüssigkeit von dem am Glase angeschossenen kohlensauren Kalk abgegossen und letzterer abgespült werden konnte. Er wurde sodann in Salzsäure gelöst, mittelst Zusatz von Schwefelsäure in Gips verwandelt, in einem gewogenen Platintiegel abgeraucht und geglüht. Aus der Menge des Gipses wurde die der Kohle berechnet; auf das Ganze betrug sie 0,02586 Grm.

Nach Verbrennung der Kohle in Sauerstoffgas blieben 0,00525 Grm. Chromeisen zurück, welches Zinnoxid enthielt.

Das in Flusssäure Aufgelöste hatte gegeben:
\begin{center}
\begin{tabular}{ l r }
    Talkerde & 0,0050\\
    Eisenoxydul & 0,0266\\
    Nickeloxyd & 0,0055\\
    Tonerde & 0,0025\\
    Zinnoxyd & 0,0020\\
    Kieselerde & 0,0462\\
\end{tabular}
\end{center}
\paragraph{}
Die Kieselerde ist aus dem Verlust hergeleitet. Kalk fehlte ganz. Die Talkerde hielt eine Spur von Manganoxydul, das Nickeloxyd eine von Kobalt. Klar ist, dass das unlösliche Mineral im Meteorstein von Alais nicht gleicher Art ist mit dem in den vorhergehenden.

Dieser Meteorstein kann für nichts anderes als für einen Erdklumpen gehalten werden, und zeigt, dass die Bergarten in seiner Heimat durch einen geologischen Prozess in Erde verwandelt wurden, wie es auf unsern Planeten der Fall ist. Der Umstand, dass darin metallisches Eisen, Schwefeleisen, nebst den Oxyden von Nickel, Kobalt, Zinn, Kupfer und Chrom enthalten sind, zeigt, dass diese Erde aus der gewöhnlichen. Meteorsteigmasse, welche hier hauptsächlich aus Meteor-Olivin bestand, gebildet worden ist. Es leidet folglich keinen Zweifel, dass der untersuchte Stein, ungeachtet aller seiner Verschiedenheiten im Äußern, ein Meteorstein ist, welcher, aller Wahrscheinlichkeit nach, aus der gewöhnlichen Heimat der Meteorsteine herstammt.

Der Kohlengehalt darin scheint ursprünglich nicht bloß Kohle gewesen zu sein; dies sieht man am besten daran, dass das Steinpulver eine ins Grüne fallende bräunliche Farbe besitzt, aber bei der trocknen Destillation kohlschwarz wird. Die Kohle befindet sich also in einer Verbindung, welche in der Hitze zersetzt wird unter Zurücklassung von Kohle und Entwicklung von Kohlensäuregas, entweder allein oder in Begleitung von Wasser. Im ersten Fall befindet sich die Kohle bloß mit Sauerstoff verbunden, zu einem der Honigsteinsäure ähnlichen Körper, im letzten Fall aber in Verbindung mit Sauerstoff und Wasserstoff. Indes ist ein solcher Körper, der nur in Kohle, Kohlensäure und Wasser zerfällt, noch nicht bekannt. Mehr Analogie mit tellurischen organischen: Verbindungen hat der Stoff, den das Wasser zugleich mit Bittersalz auszieht. Die Anwesenheit eines kohlenhaltigen Stoffs in der Meteorerde hat Analogie mit dem Humusgehalt der tellurischen Erde; aber er ist vermutlich auf eine andere Weise hinzugekommen, hat andere Eigenschaften, und scheint nicht zu der Vermutung zu berechtigen, dass er eine analoge Bestimmung habe, wie die kohlenhaltigen Stoffe in der tellurischen Erde.

Das eben mitgeteilte Resultat zeigt mit dem von Thénard erhaltenen einige Verschiedenheiten. Indes beweist doch nichts, dass wir nicht denselben Stoff untersucht haben. Schon der Umstand, dass ich vor der Analyse 10 Prozent lösliche Salze, gemengt mit einem organischen Stoff, und 12 Prozent dem Magneten folgsame Teile abschied, bildet einen wesentlichen Unterschied. Thénard suchte Tonerde darin, ohne sie zu finden. Diess ist gewöhnlich bei Mineralien, welche Talkerde enthalten, wenn der mit ätzendem Ammoniak erhaltene Niederschlag mit Kali behandelt wird. Wiederauflösung in Säure und Fällung mit einen doppelt-kohlensauren Salz scheint von Thénard nicht angewandt worden zu sein.

\section{Pallas-Eisen und Pallas-Olivin.}
\paragraph{}
Diese berühmte Meteormasse, welche durch Pallas in Europa bekannt worden ist, lag auf dem Kamm eines Schieferberges in einer Gegend von Sibirien, zwischen Krasnojarsk and Abekansk. Die Einwohner sahen sie für ein vom Himmel gefallenes Heiligtum an, und die Volkssage bewahrte das Andenken an diesem Falle auf, wogegen alle historischen Nachrichten darüber fehlten. Pallas schätzte ihr Gewicht auf 1600 Pfund. Gegenwärtig möchte sie wohl ganz und gar unter die Öffentlichen und privaten Mineralienkabinette verteilt sein. Diese ungewöhnliche Meteormasse bestand hauptsächlich aus einem Skelett von Eisen, ähnlich einem wohl ausgegorenen Brot, dessen runde und dichte Höhlungen ausgefüllt waren mit jenem grünlichen glasklaren Olivin, dessen in dem Vorhergehenden erwähnt wurde.

Das zur Analyse angewandte Pallas-Eisen wurde erst gehämmert, so dass aller daran festsitzende, oft nicht sichtbare Olivin zerstoßen wurde und abfiel. Darauf wurde es durch etwas verdünnte Schwefelsäure vom Rost gereinigt, wohl abgewaschen und in einer Temperatur über 100° C. getrocknet. Nun wurde es in Salzsäure gelöst, und das Wasserstoffgas durch eine mit Ätzeammoniak versetzte Lösung von salpetersaurem Silberoxyd geleitet. Anfangs trübte sich diese Flüssigkeit nicht, aber gegen das Ende, besonders als die Lösung des Eisen durch etwas Wärme unterstützt wurde, zeigten sich deutlich Spuren von Schwefelwasserstoffgas, doch durchaus zu unbedeutend, um dem Gewichte nach bestimmt a werden, wiewohl der Versuch mit mehr als 7,742 Grm. Pallas-Eisen angestellt wurde.

Als alle Gasentwicklung in der Wärme aufhörte, obschon die Flüssigkeit noch viel freie Säure enthielt, wurde das Klare abgegossen von dem Rückstand, welcher bestand teils aus einem zarten kohleähnlichen Stoff, teils aus kleinen metallisch glänzenden Körnern und Flitterchen, auf welche frische Salzsäure ohne Wirkung war. Sie wurden auf ein Uhrglas gebracht, und auf demselben gewaschen und getrocknet, damit sie bei einer späteren Probe auf einen Kohlegehalt von allen Fäserchen des Filtrierpapiers frei seien. Sie wogen 0,0371 Grm. oder 0,48 eines Prozents.

Die Eisenlösung wurde ‚mit Salpetersäure oxydiert, mit ätzendem Ammoniak vermischt bis ein großer Teil des Eisenoxyds niedergefallen war und dann in der Wärme mit bernsteinsaurem Ammoniak gefällt. Das Klare wurde abfiltriert und der Rückstand mehrmals ausgekocht mit Wasser, dem etwas Salmiak und etwas bernsteinsaures Ammoniak zugesetzt war, alsdann auf ein Philtrum gebracht und gewaschen. Das Waschwasser wurde ab gedunstet und der zuerst hindurchgegangenen Flüssigkeit hinzugefügt, darauf das Ganze in einer verkorkten Flasche mit einer Lösung von Schwefelnatrium (NaS5) vermischt, und stehen gelassen, bis die Flüssigkeit klar und rein gelb geworden war, endlich das Schwefelmetall auf ein Philtrum gebracht. Das Durchgegangene wurde mit Salzsäure zerlegt, deren Überschuss ab gedunstet und darauf die filtrierte Lösung mit einem Gemenge von Ammoniak und phosphorsaurem Natron versetzt. Sie trübte sich nicht sogleich, aber nach einer Weile schied sich ein großschuppiges Salz ab, welches phosphorsaurer Ammoniak-Talkerde glich, und, nachdem es gesammelt, gewaschen und geglüht worden, schwarz war. Es war, wie sich fand, hauptsächlich phosphorsaures Mangan, welches beim Glühen in ein basisches Oxydsalz übergegangen war. Vollkommen frei von Talkerde war es wohl nicht; allein ich habe es im Resultat als Mangansalz berechnet. Es wog 0,028 Grm., entsprechend 0,0103 Grm. oder 0,13 Prozent Mangan.

Die Schwefelmetalle, welche, damit sie nicht oxydierten, mit kochendem Wasser gewaschen worden waren, wurden sodann geröstet, in Salzsäure gelöst, die überschüssige Säure durch Abdampfung zur Trockne auf dem Wasserbade fortgeraucht, das Salz abermals in Wasser gelöst und die Lösung mit ätzendem Ammoniak vermischt, wodurch sie stark blau wurde, aber auch ein Niederschlag entstand, der sich in einem größeren Zusatz von Ammoniak nicht löste. Dieser Niederschlag wurde auf einem Philtrum gesammelt; er war schön grün und erwiess sich als unlöslich in kohlensaurem Ammoniak. Er wog geglüht 0,03 Grm., war nun schwarzgrau und zeigte sich bei einem Versuche hauptsächlich aus Kobaltoxyd bestehend.

Aus diesen Versuchen, welche besonders angestellt wurden, um die Ursache des gegen meine früher bei Behandlung von kobalthaltigem Nickel gemachten Erfahrurgen so abweichenden Verhaltens auszumitteln, ging hervor, dass, wenn die Lösung kein Ammoniaksalz enthält, mit welchem sich das Doppelsalz bilden kann, ein Teil des Kobaltoxyds mit grüner Farbe niederfällt, und, wenn die Flüssigkeit zugleich Talkerde enthält, auch diese vereinigt mit Kobaltoxyd niederfällt, und dass diese Verbindung beim Waschen grün bleibt, wogegen das Kobaltoxyd für sich braun wird. So oft die Lösung einen Überschuss von Säure oder ein Ammoniaksalz in der zur Bildung von Doppelsalzen hinreichenden Menge enthält, so wird jener Niederschlag nicht anders erhalten als wenn das Nickeloxyd mit Kalihydrat niedergeschlagen wird, wobei er dann mit diesem niederfällt, und der Kobaltgehalt so ganz aus der mit ätzendem Kali gefällten Flüssigkeit verschwunden ist, dass sich darin nicht eine Spur davon mehr vorfindet.\footnote{In Bezug auf die Bildung dieser Kobaltverbindung mag noch Folgendes angeführt sein. Reines nickelfreies Kobaltoxyd, nach Laugiers Methode dargestellt, wurde in Salzsäure aufgelöst und die Lösung im Wasserbade zur Trockne abgedampft. Das blaue Salz wurde in Wasser gelöst. Mit ätzendem Ammoniak gab es einen grünen Niederschlag, welcher nach einigen. Stunden braun ward. Ein anderer Teil des Salzes wurde mit etwas Chlormagnesium vermischt; es gab mit Ammoniak ebenfalls einen grünen Niederschlag, aber dieser wurde nicht braun. Diese Niederschläge, sowohl das reine grüne Oxyd als das talkerdehaltige, lösten sich ohne Rückstand sogleich wieder auf, als eine Lösung von Salmiak hinzugesetzt wurde. Die Lösung würde nicht rot, sondern schmutzig gelb. Ätzendes Kali, in hinreichender Menge zugesetzt, fällte das Oxyd wieder mit grüner Farbe. Das reine ward braun, das talkerdehaltige hielt sich in der Flüssigkeit unverändert grün, beinahe eine ganze Woche lang. Es war dem, welches man unter den gewöhnlichen Umständen vom Nickeloxyd erhält, so gleich, dass es durch das bloße Ansehen nicht von diesem unterschieden werden konnte. Jedoch enthielt es nicht mehr als knapp 10 Prozent Talkerde, und folglich viel freies Kobaltoxyd. Die darüberstehende Flüssigkeit war farblos. Hieraus ersieht man leicht, dass die Philips‘sche Methode bei Analysen dieser Art leicht irreführen kann, und dass ein auf diese Weise bestimmter Kobaltgehalt wohl niemals vollkommen sicher sein kann; was auch von dem hier Mitgeteilten gilt. Es ist jedoch für den hier in Rede stehenden Fall von keiner Wichtigkeit, ob dabei ein geringer Fehler vorhanden sei. Der oben angeführte Versuch beweist, dass das Schwefelmagnesium desselben Neigung hat, sich mit dem Schwefelkobalt und Schwefelnickel niederzuschlagen, wie die Talkerde mit den Oxyden dieser Metalle. Ich glaube, dass diese Umstände bei der Analyse von Verbindungen, die Nickel und Kobalt enthalten, Beachtung verdienen.}

Die oben angeführten 0,03 Grm., auf angegebene Weise mit verdünnter Salpetersäure, und diese Lösung mit Schwefelwasserstoff-Schwefelammonium behandelt, wurden zerlegt in 0,00625 Grm. Talkerde, verunreinigt mit ein wenig Mangan, aber doch die blaue Farbe auf einem geröteten Lackmuspapier wiederherstellend, und in 0,02375 Grm. Kobaltoxyd, in welchem sich eine geringe Menge Nickeloxyd entdecken ließ.

Die blaue ammoniakalische Flüssigkeit mit ätzendem Kali gefällt, gab ein schön apfelgrünes Nickeloxyd, welches geglüht 1,02175 Grm. wog. Um seinen Sauerstoffgehalt zu erproben, wurden 0,981 Grm. davon durch Glühen in einem Strom Wasserstoffgas reduziert, und dadurch ein silberweißes Metall erhalten, 0,771 Grm. wiegend. Dies hätte, nach dem Sauerstoffgehalt des gewöhnlichen Nickeloxyds berechnet, 077213 wiegen müssen, jenes enthielt folglich eine geringe Einmengung von Superoxyd. Das erhaltene Quantum Nickeloxyd, gemäß der Reduktionsprobe auf Metall berechnet, entspricht 0,803 Grm. oder 10,372 Prozent metallischen Nickels. Es wurde unter Anwendung von Wärme in Salzsäure gelöst, und die Lösung mit Schwefelwasserstoff gefällt; der gelbe und nach dem Trocknen fast schwarze Niederschlag wog geglüht, 0,002 Grm. und war kupferhaltiges Zinnoxid.

Die Flüssigkeit, aus welcher das Nickeloxyd mit ätzendem Kali gefällt worden, hatte einen deutlichen Stich ins Rosenrote. Sie gab nach dem Fortdunsten des Ammoniaks Kobaltoxyd, welches geglüht 0,021 Grm. wog, was, nebst. den zuvor erhaltenen 0,02375, zusammen 0,04475 Oxyd oder 0,03521 Kobalt ausmacht; letzteres entspricht 0,455 Prozent vom Gewicht des Meteoreisens.

Um zu untersuchen, ob das Eisen Kohle enthalte wovon ein Teil mit dem Wasserstoffgase fortgegangen sein konnte, wurden 6,132 Grm. Pallas-Eisen mit Hülfe von Wärme in verdünnter destillierter Schwefelsäure aufgelöst. Das Wasserstoffgas wurde durch eine mit Kupferoxyd gefüllte und über der Weingeistlampe erhitzte Glaskugel geleitet, wodurch es, nachdem die atmosphärische Luft des Gefäßes ausgetrieben worden, in Wasser verwandelt wurde, so dass nur ganz wenig übrigblieb. Dieses wurde auf zuvor genannte Weise in ein Gemenge von Ammoniak und Chlorcalcium geleitet; allein die Menge desselben war so gering, dass der endlich herauskristallisierte kohlensaure Kalk nicht mehr als 0,03 Grm. Gips gab, entsprechend 0,00266 Grm. oder 0,043 Prozent Kohle.

Durch die in Schwefelsäure erhaltene Lösung wurde ein Strom Schwefelwasserstoff geleitet; es entstand dadurch nach einigen Augenblicken eine blassgelbe Trübung, welche nach vollständiger Ausfällung und Sammlung dunkelgelb ins Braune fallend war, und, nach Fortbrennung des Schwefels, 0,005 Grm. Zinnoxid zurück liefs, so stark aber mit Kupferoxyd verunreinigt, dass es im geglühten Zustand fast schwarz war, und bei Reduktion eine Zinnkugel gab, deren Farbe sichtlich ins Gelbe fiel. Das Oxyd entspricht 0,066 Prozent kupferhaltigen Zinns.

Der Eisengehalt wurde nach dem Grundsatz bestimmt, dass das, was nichts anders war, Eisen sein musste. Das mit Bernsteinsäure verbunden erhaltene Oxyd wurde geprüft: a. durch Schmelzen mit Salpeter und etwas kohlensaurem Alkali auf einen Chromgehalt. Es konnte aber davon keine Spur entdeckt werden. Die Salpeterlösung mit Bleisalz versetzt, wurde zwar gelb durch Säuren, aber nur von salpetrigsaurem Blei, welches sich nicht niederschlug, sondern aufgelöst blieb. b. Nach Auflösung in Salzsäure und Fällung mit Schwefelnatrium wurde die rückständige Flüssigkeit auf einen Gehalt an Phosphorsäure geprüft, wovon indes keine deutliche Spur entdeckt werden konnte.

Zufolge dieser Untersuchung besteht das Pallas-Eisen aus:
\begin{center}
\begin{tabular}{ l r }
    Eisen & 88,042\\
    Nickel & 10,732\\
    Kobalt & 0,455\\
    Magnesium & 0,050\\
    Mangan & 0,132\\
    Zinn und Kupfer & 0,066\\
    Kohle & 0,043\\
    Schwefel & Spur\\
    Unlöslichem Rückstand & 0,480\\
    & 100,000\\
\end{tabular}
\end{center}
\paragraph{}
Klaproth gibt an, das Pallas-Eisen enthalte nur zwei Prozent Nickel und löse sich ohne Rückstand. Howard schloss aus seinen Versuchen, der Nickelgehalt betrage 17 Prozent. Der Gehalt an Magnesium, ungeachtet er nicht ungewöhnlich im Gusseisen ist, könnte der Gegenwart von Olivin zugeschrieben werden; allein wenn dies der Fall gewesen wäre, hätte immer die Kieselerde des Olivins in weißen ganz sichtbaren Körnern unter den ungelösten schwarzen oder metallischen Rückstand vorhanden sein müssen, wogegen das olivinfreie Eisen nicht die geringste Spur von Kieselerde ungelöst liefs und doch Magnesium enthielt. Wir werden überdies bei der Analyse des unlöslichen Rückstands einen neuen Beweis dafür erhalten, dass Magnesium metallisch im Meteoreisen enthalten ist.

Dieser unlösliche Rückstand ist ein ganz interessanter Teil des Meteoreisens. Er ist ganz dieselbe Phosphorverbindung, welche ich bei Untersuchung des Meteoreisens von Bohumilitz analysiert und beschrieben habe.\footnote{Annalen, Bd. XXVII, S. 126.} So wie er nach der Auflösung des Eisens zurückblieb, bestand er aus im Äußern verschiedenen Teilen, von welchen der eine schwarz, kohleähnlich und leicht, der andere metallisch glänzend und kristallinisch war. Ich hielt den ersteren für Kohle, und sonderte deshalb von ihm so viel ab als nötig war, um ihn, wie einen kohlehaltigen Stoff, in Sauerstoffgas zu verbrennen und die Kohlensäure aufzufangen. Wirklich trat auch eine ganz lebhafte Verbrennung ein; allein das Pulver nahm dabei bedeutend an Gewicht zu, und ich erhielt nur eine Spur von Kohlensäure. Die Masse war nichts anderes als dieselbe Verbindung wie der kristallinische Teil, nur so mit Eisen gemengt, dass sie gestaltlos und ungemein fein zerteilt war.

Das Metallische zeigte sich unter dem Mikroskop als bestehend aus Krystallen, welche die Eigentümlichkeit besaßen, dass sie an einigen Seiten vollkommen auskristallisiert waren, während andere durchaus Bruchflächen glichen. Seine Farbe war ganz die des Meteoreisens. Es wurde nicht von Salzsäure angegriffen, wohl aber von Königswasser, worin es sich mit Leichtigkeit löste. Ich hatte nur 0,03 Grm. davon zur Analyse zu verwenden. Dieses Quantum wurde im Königswasser gelöst, die Lösung mit ätzendem Ammoniak neutralisiert, mit Schwefelwasserstoff-Schwefelammonium in Überschuss vermengt, dieser Überschuss durch Kochen verjagt, darauf der Niederschlag von Schwefelmetallen auf ein Philtrum gebracht und mit siedend heißem Wasser gewaschen. Die durchgegangene Flüssigkeit wurde durch Abdunsten eingeengt und in einer Flasche mit ätzendem Ammoniak vermischt; es entstand kein Niederschlag, zum Beweise der Abwesenheit von Talkerde. Nun wurde Chlorcalcium hinzugesetzt, solange noch ein Niederschlag entstand, die Flasche verkorkt und die Flüssigkeit zum Klären hingestellt. Der abgesetzte phosphorsaure Kalk, welcher geglüht 0,023 Gran. wog, gab vor dem Lötrohr reichlich Phosphoreisen und entsprach 0,0049 Grm. Phosphor.

Der Niederschlag von Schwefelmetallen, wurde mit Königswasser oxydiert, die Flüssigkeit mit ätzendem Ammoniak neutralisiert, und das Eisen mit bernsteinsaurem Ammoniak ausgefällt; dies gab 0,021 Grm. Eisenoxyd, entsprechend 0,01456 Grm. Eisen. Die mit bernsteinsaurem Ammoniak gefällte Lösung wurde blau von Ammoniak, ohne gefällt zu werden. Ätzendes Kali fällte einen flockigen, voluminösen, blassgrünen Niederschlag, welcher deutlich noch etwas mehr als Nickeloxyd enthielt. Er wog geglüht 0,01175. Behandelt auf zuvor angeführte Weise mit verdünnter Salpetersäure und Schwefelwasserstoff-Schwefelammonium, wurden daraus 0,00475 Grm. weißer Talkerde erhalten, welche gerötetes Lackmuspapier stark wieder bläute. Sie entspricht 0,00191 Grm. Magnesium, Das Gewicht des Nickeloxyds betrug also 0,007 Grm., entsprechend 0,0055 Grm. Nickel. Es enthielt eine Spur von Zinn, Kupfer und Kobalt, welche, wenn die Probe mit einer größeren Menge der Masse angestellt worden wäre, sich gewiss dem Gewichte nach hätte bestimmen lassen, was aber so im Kleinen nicht anging.

Der mit Schwefelwasserstoff-Schwefelammonium erhaltene Niederschlag enthielt nicht nur Schwefelmetalle, sondern auch eine Portion phosphorsaurer Ammoniak-Talkerde. Ein Teil dieser Phosphorsäure hatte sich unzweifelhaft zugleich mit dem bernsteinsauren Eisenoxyd niedergeschlagen, in welchem aber bei Anstellung des Versuchs die Nachsuchung nach dieser Säure vergessen worden, was ich nun nicht wieder gut machen konnte. Ein anderer Teil fand sich in dem Ätzkali, welches zur Abscheidung des Nickeloxydes gedient hatte, und wurde durch Verwandlung des Salzes in Chlorkalium, und durch Fällung mit Chlorcalcium und ätzendem Ammoniak daraus abgeschieden. Sie gab 0,0035 Grm. phosphorsauren Kalk, entsprechend 0,00064 Grm. Phosphor; zusammen waren es also 0,00554 Grm. oder 18,47 Prozent Phosphor.

Der Versuch hatte folglich gegeben:
\begin{center}
\begin{tabular}{ l r r }
    Eisen & 0,0146 & 48,67\\
    Nickel & 0,0055 & 18,33\\
    Magnesium & 0,0029 & 9,66\\
    Phosphor & 0,0055 & 18,47\\
    Verlust & 0,0015 & 4,87\\\hline
    & 0,0300 & 100,00\\
\end{tabular}
\end{center}
\paragraph{}
Der Verlust ist ein wenig groß, aber bei der geringen Menge der Probe leicht erklärlich. Ein kleiner Teil davon ist Kohle. Nähme man den ganzen Verlust für Phosphor, so würde das Verhältnis nahe R2P; allein ein so im Kleinen angestellter Versuch mit einem so zusammengesetzten Körper kann nicht genau genug sein, um mit Sicherheit darauf eine Rechnung zu begründen.

Bei diesen Versuchen fand ich, dass das mit Unterstützung von Wärme in einer etwas verdünnten Säure aufgelöste Pallas-Eisen, nachdem die Flüssigkeit sich stark mit einem neutralen Eisensalze gesättigt hatte, ein Skelett von der Form des Eisens zurückließ; das jedoch so leicht war, dass es von dem sie entwickelnden Gase in der Flüssigkeit herumgeführt wurde. Ich liefs die Flüssigkeit stehen bis alle Gasentwicklung aufgehört hatte und wusch dann das Skelett mit siedendem Wasser aus. Es war schwarz und so porös, dass es zwischen den Fingern zusammengedrückt werden konnte. Zur Analyse konnte ich davon nur 0,088 Grm. anwenden. Es ließ sich in Sauerstoffgas verbrennen, brannte mit Lebhaftigkeit und erzeugte 3,75 Milligrm. kohlensauren Kalk, während sein Gewicht bis auf 0,114 sich vermehrte. Es wurde in Salzsäure gelöst und nach dem zuvor beschriebenen Plan analysiert; dadurch wurden erhalten:
\begin{center}
\begin{tabular}{ l r }
    Eisen & 57,18\\
    Nickel & 34,00\\
    Magnesium & 4,52\\
    Zinn und Kupfer & 3,75\\
    Kohle & 0,55\\
     & 100,00\\
\end{tabular}
\end{center}
\paragraph{}
Es enthielt eine äußerst geringe Spur von Phosphor, welche indes wahrscheinlich den vom Skelett noch umschlossenen Teilen der Phosphorverbindung angehörten. Die Gegenwart vom Magnesium hierin beweist, dass dieses Metall in Verbindung mit Nickel und Eisen weniger löslich ist als selbst das Eisen.

Der Pallas-Olivin ist von Walmstedt\footnote{Kongl. Vetensk. Acad. Handl. f. 1824 p. 361 (Ann. Bd. IV, S. 198).} und von Stromeyer\footnote{Götting. gelehrt, Anzeigen, d. 27. Dez. 1824 (Annal. Bd. IV, S. 193).} untersucht worden. Ersterer fand, dass die Zusammensetzung dieses Minerals sich vollkommen durch die Formel {Mg,f}S ausdrücken lasse. Letzterer, welcher Nickel in anderen Olivinen gefunden, fand, wider alle Vermutung, dass der Pallas-Olivin frei davon sei, wiewohl schon Howard angegeben, dass darin bis zu 1 Prozent Nickeloxyd vorkomme. Hr. Prof. Walmstedt hat die Güte gehabt, mir eine kleine Probe von dieser, nunmehr seltenen Substanz abzulassen; sie war aus der größeren Stufe vom Pallas-Eisen, die im Mineralienkabinett der Universität zu Uppsala aufbewahrt wird, herausgefallen.

Ich habe sie nach dem zuvor beschriebenen Plan analysiert, d. h. mittelst Zersetzung des geschlemmten Pulvers durch Salzsäure, Behandlung der Lösung mit Schwefelwasserstoffgas u. s. w.,\footnote{Es dienten hierzu Salzsäure und Wasser, die zuvor mit Schwefelwasserstoff gesättigt und durch Kochen wieder davon befreit waren, um der Abwesenheit eines jeden Zinngehalts der Reagenzien sicher zu sein.} und erhielt dabei ein mit Kupferoxyd verunreinigtes Zinnoxid, ganz wie aus den vorhergehenden Meteorsteinen, konnte aber darin, wie Stromeyer, nicht die geringste Spur von Nickel entdecken. Meine Analyse stimmte übrigens fast vollkommen mit der von Walmstedt überein. Er fand eine Spur von Kalk und Tonerde, ich eine von Kali und Natron, welche zusammen 0,007 Prozent Chloralkalium gaben.

Ich werde hier die Resultate meiner und Walmstedts Analyse zusammenstellen:
\begin{center}
\begin{tabular}{ |p{30mm}|p{20mm}|p{20mm}|p{26mm}| }
    \hline
      & W. & B. & Sauerstoffgehalt\\\hline
      Kieselerde & 40,83 & 40,86 & 21,039\\\hline
      Talkerde & 47,74 & 47,35 & 18,32\\\hline
      Eisenoxydul & 11,53 & 11,72 & 2,67\\\hline
      Manganoxydul & 0,29 & 0,43 & 0,09\\\hline
      Zinnoxid & -,- & 0,17  & \\\hline
       & 100,39 & 100,53 & \\
    \hline
\end{tabular}
\end{center}
\paragraph{}
Die Gegenwart von Zinnoxid in diesem Olivin veranlasste mich nachzusehen, ob dasselbe Oxyd auch in tellurischen Olivinen vorkomme, deren Nickelgehalt sie früher in Übereinstimmung mit den meteorischen gebracht hatte. Zu dieser Untersuchung wählte ich zwei Arten von Olivinen, von denen die eine in Böhmen, bei Boskovic, nicht weit von Aussig, vorkommt, und die andere von mir selbst aus einer der Lavamassen im Departement Puy de Dome herausgelöst worden war. Beide enthielten, ganz wie die zuvor untersuchten Meteorsteine, Zinnoxid, verunreinigt mit Kupferoxyd, in Menge nicht voll 0,2 Prozent betragend. Die Anwesenheit des Kupferoxyds entdeckte sich mit Leichtigkeit als das Zinnoxid vor dem Lötrohr auf Kohle mit ganz wenig Borax geschmolzen wurde; das Zinn reduzierte sich dabei zu einer geflossenen Kugel, und liefs an der Seite ein Glas zurück, welches beim Erkalten undurchsichtig und rot wurde.

Der böhmische Olivin war im Ansehen vollkommen dem Pallas-Olivin gleich. Ich setzte deshalb die Analyse desselben fort, um zu finden, ob er nickelfrei sei. Die Lösung wurde mit Salpetersäure oxydiert, das Eisenoxyd mit bernsteinsaurem Ammoniak gefällt und die filtrierte Auflösung mit kohlensaurem Ammoniak so gesättigt, dass eine schwache alkalische Reaktion, aber kein Niederschlag entstand; nun fällte Schwefelwasserstoff-Schwefelammonium eine Portion Schwefelnickel, welches, mit Phosphorsalz und metallischem Zinn geprüft, eine Spur von Kobalt entdecken ließ. Diese Übereinstimmung zwischen dem tellurischen und meteorischen Olivin im Gehalt von zufälligen Bestandteilen ist meiner Meinung nach besonders merkwürdig.

\section{Meteoreisen von Elbogen.}
\paragraph{}
Über den Fall dieser Meteoreisen-Masse ist keine historische Nachricht vorhanden; allein ihre Aufbewahrung seit unbekannter Zeit auf dem Rathause der Stadt Elbogen deutet darauf hin, dass ihr Fall beobachtet worden ist, und dies Veranlassung gegeben hat, sie in Sicherheit zu bringen. Der ihr vom Volke daselbst gegebene Name der verwünschte Burggraf scheint darauf hinzuweisen, dass sie innerhalb des ziemlich kurzen Zeitraums, da Elbogen von Burggrafen regiert wurde, was zu Ende des 14. und zu Anfang des 15. Jahrhunderts geschah, niedergefallen ist. Sie wird jetzt in Wien aufbewahrt. Das Stück, welches ich davon besitze, ist sicher durch viele Hände gegangen, ehe es in die meinigen kam, und sein Ursprung von der Elbogener Masse kann also nicht als ganz sicher angesehen werden; allein die darauf durch Ätzung hervorgerufenen Figuren stimmen so überein mit dem Abdruck von den Figuren des Wiener Stücks, welche in v. Schreibers Beiträgen zur Geschichte und Kenntnis meteorischer Stein- und Metallmassen Taf. X mitgeteilt werden, dass ich keinen Grund zum Argwohn einer Vertauschung habe, besonders da so solide Stücke von Meteoreisen nicht gemeint sind.

Zur Analyse wurde ein abgesägtes Scheibchen von 1,47 Grm. Gewicht angewandt. Es wurde in Salzsäure gelöst. Das Wasserstoffgas gab in ammoniakhaltiger Silberlösung eine äußerst geringe aber doch nicht zu verkennende Spur Schwefel. Während der Lösung fiel von dem reineren Eisen ein rußähnliches Pulver ab, was auf den Widmanstädten’schen Figuren nicht bemerkt wurde, da diese sich blank erhielten. Außer diesem schwarzen Pulver fielen, wie vom Pallas-Eisen, metallische Flitterchen ab, von denen jedoch einige ziemlich groß waren, und zerbrochenen Krystallen glichen. Das Ungelöste wog 0,0325 Grm. oder 2,211 Prozent.

Die Lösung wurde auf gleiche Weise wie die des Pallas-Eisens analysiert. Das Elbogener Eisen fand sich darnach bestehend aus:
\begin{center}
\begin{tabular}{ l r }
    Eisen & 88,231\\
    Nickel & 8,517\\
    Kobalt & 0,762\\
    Magnesium & 0,279\\
    Phosphormetallen & 2,211\\
    Schwefel und Mangan & Spur\\
     & 100,000\\
\end{tabular}
\end{center}
\paragraph{}
Das Nickel enthielt Zinn und Kupfer, aber ich hatte nicht Material genug zu einer Bestimmung ihrer Menge, die jedenfalls höchst gering war. Klaproth fand im Elbogener Eisen nur 2 1/2 Prozent Nickel, Neumann dagegen 6,45 Proz. Der Anblick der verschiedenen Dichtheit der Figuren auf diesem Meteoreisen beweist, dass es in verschiedenen Stücken nicht gleich zusammengesetzt sein könne; eine solche Abweichung aber, als sich in Klaproths Resultate findet, kann nur die Folge einer fehlerhaften Methode in Abscheidung des Nickels sein.

Die unlöslichen Phosphormetalle gleichen ganz denen aus dem Pallas- und Bohumilitz-Eisen, stimmen aber in ihren Zusammensetzungsverhältnissen am nächsten mit denen aus letzterem. Ich erhielt nämlich, einen Verlust ungerechnet, der fast eben so groß war wie bei den Flitterchen aus dem Pallas-Eisen, bei einer Analyse von 0,028 Grm.:
\begin{center}
\begin{tabular}{ l r }
    Eisen & 68,11\\
    Nickel und Magnesium & 17,72\\
    Phosphor & 14,17\\
\end{tabular}
\end{center}
\paragraph{}
Die Phosphormetalle des Bohumilitz-Eisens gaben 65,977 Eisen, 15,008 Nickel, 14,023 Phosphor, 2,037 Kiesel, 1,422 Kohle. Vom Kiesel fand sich keine Spur im Elbogener Eisen; auf Kohle wurde es nicht untersucht, da dazu nur unverbrennliche Krystalle angewandt wurden. Das Nickel enthielt eine Spur von Zinn und Kobalt. Übrigens wurde hier derselbe Fehler wie bei Untersuchung der Phosphorverbindungen des Pallas-Eisens begangen, nämlich in dem mit bernsteinsaurem Ammoniak abgeschiedenen Eisenoxyd die Phosphorsäure nicht nachgesucht. Die mit dem Bohumilitz-Eisen angestellten Versuche ergaben, dass sich wirklich Phosphor darin befindet; außerdem, was in die unlöslichen Phosphormetalle eingeht. Vermutlich enthalten die letzteren auch dort Magnesium.

Die nun angeführten Untersuchungen zeigen, dass die Meteorsteine Bergarten sind, gemengt aus mehreren Mineralien in variierendem Verhältnis. Diese Mineralien sind folgende:

1. Gediegenes Eisen. Diess macht zuweilen die Hauptmasse des Niedergefallenen aus, doch ist, soweit bisher bekannt, seit 1802 keine solche Masse gefallen. Die Meteorsteine, in denen Eisen der überwiegendste Bestandteil ist, zerspringen nicht beim Fall, und bilden daher die größten der gefundenen Meteorsteine. Das Eisen darin bildet zuweilen eine dichte Masse, zuweilen gewundene kleinere und größere Teile, so wie Körner, gewöhnlich voller Grübchen und Höhlungen, welche eine Steinmasse umschließen. Das Eisen ist gemengt mit anderen Metallen, hauptsächlich mit Nickel, dessen Quantität nicht beständig zu sein scheint. In dem übriges ist eine chemische Verbindung von Eisen und Nickel angeschossen, und da sie sich träger in Säuren löst als das dazwischen befindliche reinere Eisen, so entstehen durch Ätzung die unter dem Namen der Widtmanstädten'schen Figuren bekannten Zeichnungen von diesen Krystallen. Lässt man eine solche geätzte Oberfläche nach dem Polieren anlaufen, so wird das Eisen dunkelblau und die Nickellegierung brandgelb. Das Eisen enthält überdies kleine Quantitäten von Kobalt, Magnesium, Mangan, Zinn, Kupfer, Schwefel und Kohle, zuweilen auch eine Spur von Phosphor. Schwefel und Kohle gehen mit dem Wasserstoffgase fort. Zinn und Kupfer lösen sich, selbst ohne Zusatz von Salpetersäure, neben dem Eisen und Nickel Wenn das gediegene Eisen aufgelöst wird, fallen in Säuren unlösliche Phosphormetalle von ihm ab; ein Teil derselben war gleichförmig mit dem Eisen gemengt und scheidet sich in leichten schwarzen, Kohle ähnlichen Flocken ab, ein anderer Teil dagegen in kleinen metallisch glänzenden Krystallen, welche die Eigentümlichkeit zeigen, dass einige ihrer Seiten wie Bruchflächen aussehen, die anderen ächte Krystallflächen sind. Das schwarze Pulver verbrennt in Sauerstoffgas and gibt dabei eine geringe Spur von Kohlensäure. Die eigentliche Verbrennung gehört den Metallen und dem Phosphor an. Ohne Zweifel sind die Krystalle Phosphorite von Eisen, Nickel und Magnesium in bestimmtem Verhältnisse; aber meine Versuche darüber sind noch so unvollkommen, dass sich die richtige Zusammensetzung dieser Verbindungen nicht daraus ableiten lässt. Der erste Schritt dazu ist der gewesen, zu bestimmen, dass solche Phosphorverbindungen wirklich im Meteoreisen vorhanden sind. Wenn hinreichendes Material zu einer genauen Untersuchung zu bekommen ist, wird der zweite darin bestehen, die Zusammensetzung dieser Verbindungen genau zu bestimmen.

2. Schwefeleisen. Diels ist kein Schwefelkies, knapp Magnetkies, sondern vermutlich ein Schwefeleisen, welches 1 Atom von jedem Bestandteil enthält. Daraus erklärt sich seine geringe magnetische Polarität, so wie die große Heftigkeit, mit der es unter Entwicklung von Schwefelwasserstoffgas von Säuren zersetzt wird. Bei den von mir untersuchten Meteorsteinen habe ich es nicht in abgesonderten Teilen angetroffen, sondern so gemengt mit der Masse, dass ich durch das Ansehen derselben keine Kenntnis von ihm erhalten könnte. Es trägt vermutlich zur dunkeln Farbe der Meteorsteine bei. Es kann, wie schon Howard vermutete, niemals vollständig mit dem Magneten ausgezogen werden, weil es sich beim Zerreiben an die Teile des härteren Pulvers setzt und dieselben färbt. Howard analysierte das Schwefeleisen aus dem Meteorstein von Benares, und bekam 10,5 Eisen, Nickel 1,0, Schwefel 2,0 und Verlust 0,5. Der Schwefelgehalt ist nicht ein Drittel von dem, was Eisen und Nickel in Fe und Ni aufnehmen, und dies beweist hinreichend, dass Howard ein Gemenge von Schwefeleisen mit fein verteiltem Nickeleisen zu seiner Analyse anwandte. Solche Gemenge aber habe ich wenigstens immer erhalten, wenn ich das Schwefeleisen zu einer Analyse absondern wollte. Wie wahrscheinlich es auch sein mag, dass das Schwefeleisen ein wenig Schwefelnickel und Schwefelkupfer einschließt, so kann dies doch aus den Versuchen, die wir bisher besitzen, nicht gefolgert werden. Eine Analyse des Schwefeleisens der Meteorsteine ist folglich ein Desiderat, aber es wird dabei notwendig, sich nicht durch eine Einmengung von fein verteiltem Nickeleisen irre führen zu lassen.

3. Magneteisenstein. Wiewohl das Eisen in den Meteorsteinen hauptsächlich als Metall und im Minimo der Oxydation vorkommt, so findet sich doch ganz bestimmt Eisenoxyduloxyd in dem Meteorstein von Lontalax, deren einziger dem Magneten folgsamer Bestandteil daraus besteht, so wie in dem Meteorstein von Alais, wo es das Meiste von dem mit dem Magneten Ausziehbaren ausmacht, ungeachtet sich auch darin gediegenes Eisen in geringer Menge eingemengt befindet. Ob es sich für gewöhnlich in den Meteorsteinen finde, weiß ich nicht mit Sicherheit. Allein, wenn deren Pulver mit Salzsäure übergossen wird, entwickelt sich zuerst etwas Schwefelwasserstoffgas, welches nach einem Augenblick verschwindet, und sodann wird die Säure gelb, was ohne die Gegenwart von Eisenoxyd nicht möglich wäre, es sei denn es wäre Sauerstoff aus der Luft absorbiert. Es ist darnach glaublich, dass der Überschuss des Sauerstoffs der Basen des in Säuren löslichen Bestandteils der Meteorsteine herrührt von einer Einmengung von Eisenoxyduloxyd in so fein zerteiltem Zustand, dass es von dem Magneten nicht anders als sehr unvollkommen ausgezogen wird.

4. Meteor-Olivin macht ungefähr die Hälfte von dem aus, was nach Ausziehung der magnetischen Bestandteile zurückbleibt, und wird von Säuren mit Zurücklassung von Kieselerde zersetzt. Seine Formel ist ganz die des gewöhnlichen Olivins, nämlich {M f}S, worin M und f in relativer Menge variieren. Er enthält als isomorphe Substitutionen kleine Mengen von Nickeloxyd- und Manganoxydul-Silicat, auch eine Portion Zinnoxid, worin er dem tellurischen Olivin gleicht. Ob die kleinen Quantitäten von Kali und Natron, welche die Analyse ergab, ihm wesentlich angehörten, oder eine Probe von der angefangenen Zersetzung des unlöslichen Minerals waren, lässt sich für jetzt nicht entscheiden. Dasselbe gilt von dem geringen Tonerdegehalt, den man zuweilen darin antrifft. Bemerkenswert ist, dass er fast niemals Kalk enthält.

5. In Säuren unlösliche Silicate von Talkerde, Kalk, Eisenoxydul, Manganoxydul, Tonerde, Kali und Natron, in welchen der Sauerstoff der Kieselerde das Doppelte des der Basen ist. Vermutlich bilden sie mehr als ein Mineral, ein pyroxenartiges {M C f}S2und ein leucitartiges {M C N K}S2+3AS2.

Die schwarze Rinde auf den Meteorsteinen ist Folge der Schmelzbarkeit ihrer Silicate, welche auch dazu beitragen, den für sich unschmelzbaren Olivin in Fluss zu bringen.

Besonders verdient erwähnt zu werden, dass wenn die Meteorsteine aus tellurischem Olivin und Pyroxen gebildet wären, ihre Farbe grün oder, durch höhere Oxydation des Eisens darin gar kohlschwarz sein müsste, woraus zur Genüge erhellt, dass die geschmolzene schwarze Kruste erst in der Atmosphäre der Erde entstanden ist.

6. Chromeisen. Dass dieses Mineral ein so beständiger Begleiter der Meteorsteine ist, ist in Wahrheit bemerkenswert, da es immer nur in ganz geringer Menge darin gefunden wird. Die obigen Versuche zeigen, dass es unzersetzte abgeschieden wird, doch wird dabei auch immer ein Teil desselben zersetzt, dessen Bestandteile in dem abgeschiedenen Eisenoxyde, dem sie folgen, gesucht werden müssen.

7. Zinnstein. Der Zinngehalt der Meteorsteine rührt her teils von dem gediegenen Eisen, welches zinnhaltig ist, teils von einer geringen Menge Zinnoxid, welche, nebst Chromeisen, darin verteilt ist, und wie das Chromeisen bei der analytischen. Behandlung teils sich löst, teils zurückbleibt, gemengt mit dem Chromeisen. Das Zinnoxid ist kupferhaltig; ob es auch, wie das tellurische, etwas Eisen- und Manganoxydul enthalte, habe ich picht ausmitteln gekonnt.

Ein noch genaueres Studium der Meteorsteine zu dem Gesichtspunkt, von dem ich ausging, würde uns unzweifelhaft in Zukunft mit noch mehren Bestandteilen derselben bekannt machen.

Wenn wir die Meteorsteine als Proben von Bergarten betrachten und sie mit denen unserer Erde vergleichen, so zeigen sich dabei, auch wenn man den Gehalt an gediegenem Eisen ausnimmt, wesentliche Unterschiede. Der Reichtum an Talkerde, welche überall verwaltender Bestandteil ist, die Seltenheit der Kieselerde und der unbedeutende Gehalt an Silicaten von Tonerde und Alkali zeichnen die Meteor-Bergarten aus. Auf der Erde verhält es sich umgekehrt; hier ist die Kiesel erde überwiegend, und Silicate von Tonerde und Alkali sind überall die hauptsächlichsten Gemengteil. Die Talkerde kommt sparsam vor. Die Feinkörnigkeit und der geringe Zusammenhang in der Texter der Meteorsteine könnte darauf hindeuten, dass sie im geschmolzenen Zustande ausgeworfen wurden, und dass sie sich folglich mit den Produkten der tellurischen Vulkanen vergleichen lassen. Indes scheint ein: solcher Vorgang nicht stattgefunden zu haben. Wenn man die Textur eines größeren Meteorsteinstücks genau betrachtet, so findet man, dass sie gesprungen gewesen sind, und dass diese Sprünge ausgefüllt wurden mit einer andern mehrenteils dunkleren Steinmasse. Ähnliche Verhältnisse findet man in v. Schreibers zuvor zitierter Arbeit über Meteorsteine abgebildet. Diess weist auf eine langsamere und ruhigere Bildungsart hin. Dass der Olivin unter den tellurischen Vulkanprodukten und selten in andern vorkommt, beweist nicht die Notwendigkeit, dass der Olivin immer ein vulkanisches Produkt sein müsse. Er ist unschmelzbar und findet sich eingeschlossen in vulkanischen Bergarten, weil er nicht mit ihnen zusammen in Fluss treten kann. In den Meteorsteinen ist er dagegen so gleichförmig mit den übrigen Bestandteilen gemengt, dass seine Anwesenheit in diesen offenbar einen anderen Grund hat als die der Olivindrusen in der Lava und dem Basalt. Der Meteorstein von Alais beweist, dass in der Heimat der Meteorsteine, unter irgendeinem geognostischen Ereignisse, Bergarten zerfielen und sich in eine Art Erde verwandelten, und dass selbst diese olivinartige mit gediegenem Eisen gemengte Masse die Bergart war, welche zertrümmert ward. Der Gehalt dieser Erde an Salzen, die m Wasser löslich sind, scheint zu beweisen, dass jener Vorgang ohne Mitwirkung von Wasser geschah, oder in einem Wasser, welches bedeutende Mengen von diesen Salzen gelöst enthielt, so dass dieselben beim Austrocknen zurückblieben, Der kohlehaltige' Stoff, den diese Erde eingemengt enthält, scheint nicht zu dem Schlafs zu berechtigen, dass in der ursprünglichen Heimat dieser Erde eine organische Natur vorhanden sei. Diese Eigenschaft der Erde scheint mehr als ein anderer Umstand zu zeigen, dass die Meteorsteine nicht in flüssiger Form ausgeworfen wurden und sodann erkalteten, weil unter solchen Umständen eine Erdbildung nicht denkbar ist.

Das eben Angeführte gilt von der Mehrzahl der Meteorsteine, welche alle als abstammend aus einer gemeinschaftlichen Gegend betrachtet werden können. Aber unter den untersuchten Meteorsteinen haben drei eine so wesentlich verschiedene Zusammensetzung gegen die übrigen gezeigt, dass man mit Sicherheit sagen kann, sie sind nicht von demselben Ort gekommen wie jene, sondern rühren entweder van einem andern Weltkörper her, oder von einer andern Gegend auf demjenigen, der uns die übrigen zusandte. Dagegen stimmen sie unter sich so gut überein, dass man wohl vermuten kann, sie haben eine gemeinschaftliche Heimat: dies sind die, welche bei Stannern in Mähren, bei Jonzac und bei Juvenas im Frankreich gefallen sind. Der erste ist von Moser und sodann von Klaproth untersucht, die beiden andern sind es von Laugier. Sie weichen von den entsprechenden darin ab, dass sie kein gediegenes Eisen enthalten, dass sie ein Aggregat von deutlich unterscheidbaren Mineralien ausmachen, wiewohl die Gemengteil von äußerst geringem Volume sind, und dass Talkerdesilikat nur zu einer ganz unbedeutenden Quantität in ihre Zusammensetzung eingeht. Dagegen enthalten sie, außer etwas Schwefeleisen, Silicate von Kalk, Tonerde und Eisenoxydul. Auch enthalten sie Chrom. Das Verhältnis zwischen dem Sauerstoff in der Kieselerde und dem in den Basen ist solcher Art, dass der erstere mehr als eben so viel und weniger als doppelt so viel wie letzterer beträgt. Ungefähr ein Drittel ihrer Masse (Kieselerde darin nicht mit begriffen) ist, nach Laugiers Analyse des Meteorsteins von Juvenas, löslich in Säuren, woraus sich wohl vermuten lässt, dass in dem löslichen Teile die Kieselerde und die Basen gleich viel Sauerstoff enthalten, in dem unlöslichen aber, der Sauerstoff in ersterer doppelt so groß als in letzteren sei, wie.in den zuvor beschriebenen Meteorsteinen. G. Rose hat diese Art von Meteorsteinen näher untersucht, und es wahrscheinlich gemacht, dass sie Gemenge seien von Labrador und Pyroxen, nebst etwas nickelfreiem Magnetkies, der indes nach seinen Versuchen dem Magneten nicht folgt.\footnote{Poggendorffs Annalen, Bd, IV, S. 173.}

Wenn die verschiedenen Arten von Meteorsteinen aus dem Monde herstammen, so scheint es klar zu sein, dass die letztere und seltenere Art aus einer Gegend desselben herrührt, die so gelegen ist, dass die von dort ausgeworfenen Körper nicht direkt auf die Erde zufliegen wie die gewöhnlichen Meteorsteine, und dass darin der Grund ihrer Seltenheit liegt. Dass das gediegene Eisen in denselben fehlt, ist bemerkenswert, und zeigt, dass der große Gehalt an gediegenem Eisen, welcher die gewöhnlichen Meteorsteine auszeichnet, nicht allgemein verbreitet ist, und es kann die Hypothese unterstützen, dass dasselbe in einer gewissen Gegend auf dem Mond reichlicher vorkomme, und dies die Ursache sei, dass diese Gegend, vermöge des magnetischen Einflusses Abseiten der Erde, dieser unverändert zugewandt ist.

Um einen Begriff zu geben, wie verschieden in ihrer Zusammensetzung diese drei Meteorsteine von den von mir beschriebenen sind, will ich hier das Resultat von Klaproths und Laugiers Analysen anführen:
\begin{center}
\begin{tabular}{ |l|r|r|r| }
    \hline
     & Stannern\footnote{Klaproths Beiträge, Bd., V. S. 237.} & Jonzac\footnote{Ann. de chim. et de phys. T. XIII p. 441.} & Juvenas\footnote{Gilberts Annal. d. Physik, Bd. LXXI, S. 208.}\\\hline
    Kieselerde & 48,25 & 46,00 & 40,0\\\hline
    Talkerde & 2,00 & 1,60 & 0,8\\\hline
    Kalk & 9,50 & 7,50 & 9,2\\\hline
    Eisenoxydul & 23,00 & 32,40 & 23,5\\\hline
    Tonerde & 14,50 & 6,00 & 10,4\\\hline
    Manganoxyd & -,- & 2,80 & 6,5\\\hline
    Kali & -,- & -,- & 0,2\\\hline
    Kupferoxyd & -,- & -,- & 0,1\\\hline
    Chromoxyd & -,- & 1,00 & 1,0\\\hline
    Schwefel & 2,75 & 1,50 & 0,5\\
    \hline
\end{tabular}
\end{center}
\paragraph{}
Der Meteorstein von Stannern ist neuerdings wieder von v. Holger\footnote{Baumgartners Zeitschrift für Physik und verwandte Wissenschaften, Bd. II, S. 293.} untersucht, aber auf eine Weise, welche keine Bürgschaft für die Richtigkeit des Resultates liefert, man kann wohl sagen, ein offenbar unrichtiges Resultat gegeben hat. Von dem, was mit Königswasser ausgezogen werden kann, glaubt er, es hätte sich als regulinisches Metall in dem Stein befunden, und auf diese Weise bekommt er einen nicht unbedeutenden Gehalt von metallischem Aluminium, Calcium, Mangan u. s. w. Wie wenig aber auch diese Untersuchung den Anforderungen unserer Zeit entspricht, so enthält sie doch eine Bemerkung, die, falls sie richtig ist, die Gegenwart eines Stoffs andeutet, welcher Moser und Klaproth entgangen ist. Als er nämlich das Eisen aus der mit Königswasser erhaltenen Lösung niederschlagen wollte, und dazu benzoesaures Alkali anwandte, war dieser Niederschlag weniger gefärbt als es mit Eisenoxydsalzen gewöhnlich ist. Durch ganz unvollkommene Versuche glaubte er gefunden zu haben, dass dieser Niederschlag, außer Eisenoxyd, Zinnoxid und Ceroxydul enthalte. Dass letzteres das nicht war, für was es v. Holger hielt, sieht man sogleich daraus, dass er es in Form von Ceroxydul abschied und abwog, da es doch unmöglich ist dasselbe isoliert darzustellen, indem es sich beim Waschen in gelbes Oxydhydrat und beim Glühen in rotes Oxyd verwandelt, in welchem Fall die Farbenveränderung daran erinnert, Oxyd statt Oxydul zu schreiben. Was das Zinn betrifft, so ist dessen Gegenwart nach dem, was ich in dem Vorhergehenden angeführt, gewiss wahrscheinlich; allein v. Holger hat das fragliche Oxyd, statt darauf die leichte und unzweideutige Probe der Reduktion zu einem Zinnkorn vor dem Lötrohr anzuwenden, aus dem Grunde für Ziem erklärt, weil es aus seiner Auflösung durch Zink. gelatinös und weiß niedergeschlagen, und durchs Glühen unlöslich wurde. Diese Eigenschaft teilt aber das Zinnoxid mit der Ton- und Zirkonerde, welche letztere zugleich, wie das Ceroxydul, von schwefelsaurem Kali gefällt wird. Auch werden diese Erden aus ihrer neutralisierten Lösung durch benzoesaures Alkali gefällt. In jeder Hinsicht verdient v. Holgers Beobachtung über die Beschaffenheit des mit benzoesaurem Alkali erhaltenen Niederschlags der Gegenstand einer abermaligen Untersuchung zu werden.

Was die einfachen Körper betrifft, welche bisher unter den Elemienten der Meteorsteine gefunden worden sind, so machen dieselben gerade ein Drittel von den auf der Erde entdeckten aus. Sie sind:

Sauerstoff als Bestandteil der Metalloxyde und Erden.

Wasserstoff, in der löslichen organischen Verbindung im Stein von Alais.

Schwefel, Bestandteil des Schwefeleisens und der schwefelsauren Talkerde.

Phosphor, in den metallischen Flitterchen, welche, bei Auflösung des Meteoreisens in Salzsäure oder Schwefelsäure, von diesem abfallen.

Kohle, im Meteoreisen und in den unbekannten Verbindungen im Stein von Alais.

Kiesel, in den Silicaten.

Von Salzbildnern ist, so weit bekannt, noch keiner gefunden. Eine Spur von Chlorverbindungen kann nach dem Falle leicht hinzugekommen sein. Von elektro-negativen Metallen ist nur das Chrom gefunden. Es wurde von Laugier entdeckt, der auch zeigte, dass es wesentlich den Meteorsteinen angehöre.

Kalium wurde zuerst von Vauquelin bemerkt, Natrium kurz darauf von Stromeyer. Calcium, Magnesium und Aluminium sind als Oxyde gewöhnliche Bestandteile der Meteorsteine.

[v. Holger hat angegeben, dass er in dem Meteoreisen von Bohumilitz metallisches Beryllium gefunden habe. Ich habe gezeigt, dass dies ein Irrtum sei, und dass er vermutlich ein Gemenge von phosphorsaurer Talkerde und phosphorsaurem Manganoxydul für Beryllerde genommen habe. Wenigstens erhielt ich nichts anderes, als ich nach Beryllerde suchte. Seitdem hat v. Holger\footnote{Baumgartners Zeitschrift: Bd. II, S. 35.} erklärt, der Irrtum liege auf meiner Seite, und der Phosphor, den ich gefunden, rühre her von einer zufälligen Verunreinigung des zur Analyse angewandten Eisens. Hierdurch veranlasst, habe ich mit einem 3,5 Grm. wiegenden Rückstand vom Bohumilitz-Eisen einen besonderen und abermaligen Versuch zur Ausziehung der Beryllerde angestellt, auf die Weise, dass ich die Lösung in eine warme Lösung von ätzendem Kali eintröpfelte und damit kochte. Die abgeschiedene ätzende Flüssigkeit hielt wirklich eine geringe Spur von Etwas aufgelöst, welches, als es gefällt war, aussah, wie eine Erde, welches aber, wie ein damit angestellter Versuch lehrte, ein Gemenge war von phosphorsaurem Kalk, Magnesia und Mangan, ohne Spur von Beryllerde.]

Von den eigentlichen Metallen: Eisen, Mangan und Nickel, entdeckt von Howard, Kobalt von Stromeyer, Kupfer von Laugier; das Zinn ist, falls das von v. Holger in dem Meteorstein von Stannern Aufgefundene etwas anders als Zinn sein sollte, zuerst in dieser Abhandlung mit Sicherheit nachgewiesen.

Zusammen sind es also achtzehn einfache Körper.
\clearpage
\end{document}
